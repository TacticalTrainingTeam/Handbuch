\newpage

\section{Die Operationsplanung}
\label{OPpla}

Zur Planung des Operationsbefehl siehe auch \ref{OPbef} 

\subsection{Auftrag}
	\begin{itemize}
		\item Was ist genau mein Auftrag?
		\item Welche Kriterien müssen erfüllt sein, dass der Auftrag erfolgreich ausgeführt wurde?
		\item Welche Aufgaben haben die anderen Einheiten?
		\item Welche Aufgaben sollen von meiner Einheit erledigt werden, welche nicht?
		\item Mit welchen Einheiten muss ich mich abstimmen für den Erfolg?
		\item Wer ist mein Ansprechpartner bei anderen Einheiten?
		\item Wer ist mein OPL?
		\item Wetter und Tageszeit
		\item Wie ist das Wetter und die Sichtverhältnisse während des Einsatzes?
		\item Wo steht die Sonne?
		\item Sind Nachtsichtgeräte notwendig?
		\item Feindkräfte (wenn unbekannt - sofort Aufklärer losschicken!)
		\item Wie sieht der Gegner aus/Wer ist der Gegner?
		\item WICHTIG: Wo befindet sich der Gegner
		\item In welcher Stärke ist der Gegner am Einsatzziel?
		\item In welcher Stärke ist der Gegner im OP-Gebiet?
		\item Wie ist der Gegner bewaffnet?
		\item Welche Anzahl an Feindfahrzeugen ist zu erwarten?
		\item Verfügt der Gegner über AT/AA oder sonstige schwere Waffen?	
		\item Verfügt der Gegner über Mörser oder sonstige indirekte Waffen?
		\item Führt der Gegner Patrouillen durch, wenn ja in welchem Radius?
	\end{itemize}

\subsection{Anfahrt}
	\begin{itemize}
		\item Wo liegt das Einsatzziel?
		\item Wie weit ist es vom eigenen Startpunkt entfernt?
		\item Wie lange brauche ich zum Angriffsziel?
		\item Welche Fahrzeuge eignen sich dazu am besten?
		\item Ist die Landezone/der Bereistellungsraum/Sammelpunkt feindfrei?
		\item Ist die Flugweg, die Straße, der Seeweg feindfrei?
	\end{itemize}

\subsection{Gelände}
	\begin{itemize}
		\item Allgemeiner Check: Wie ist das Gelände beschaffen? (Hügelig, Flach, Bewaldet, Wüste…)
		\item Kann ich meine Route so wählen, dass ich möglichst unerkannt an mein Einsatzgebiet komme?
		\item Kann ich meine Route so wählen, dass ich möglichst gute Deckung und Sichtschutz habe?
		\item Kann ich meine Route so wählen, dass ich in Sichtweite befreundeter Einheiten bleibe?
	\end{itemize}

\subsection{Einsatzziel}
	\begin{itemize}
		\item 	Wie hoch liegt das Einsatzziel?
		\item 	Wie ist es befestigt?
		\item 	Erwartet der Gegner Angriffe?
		\item 	Markiere drei mögliche Einsatzrouten
		\item 	Markiere drei mögliche LZs
		\item 	Markiere drei mögliche Verfügungräume (Sammelpunkte für Truppen vor Angriff)
		\item 	Markiere drei mögliche Aufklärer-Standorte (Sniper-Positionen)
		\item 	Markiere zwei mögliche Rückzugsrouten und Versprengtensammelpunkte
		\item 	Angriffsplanung
		\item 	Wann fand die letzte Aufklärung statt? (Ohne Aufklärung, kein Einsatz)
		\item 	Welche Aufklärungsmittel stehen mir zur Verfügung?
		\item 	Welche Fahrzeuge stehen mir zur Verfügung?
		\item 	Wieviele Mann stehen mir zur Verfügung?
		\item 	Welche Bewaffnung brauche ich aufgrund der Aufklärungsergebnisse?
		\item 	Wer agiert als Reserve? (Ohne Reserve, kein Einsatz)
		\item 	Wer übernimmt die Rolle des Medics?
		\item 	Wer übernimmt die Rolle der AT/AA
		\item 	Wird ein CAS benötigt?
		\item 	Wird ein EVAC benötigt?
		\item 	Wird ein Mörser benötigt oder sonstige indirekten Waffen?
	\end{itemize}

\subsection{Abstimmung}
	\begin{itemize}
		\item 	Welcher Rufname habe ich, welche Rufnamen die sonstigen Einheiten?
		\item 	Welche Funkkanal habe ich, welchen Funkkanal die anderen Einheiten??
		\item 	Welche Funke wird genutzt? 343, 119, etc..
		\item 	Welche Aufgaben haben meine befreundeten Einheiten?
		\item 	Welche Kartenquadranten gehören "mir"? Welche teile ich mit befreundeten Einheiten?
		\item 	Welche Maßnahmen werden ergriffen, um friendly fire zu vermeiden?
		\item 	Was passiert bei Ausfällen bei der Anfahrt?
		\item 	Was passiert mit Nachzüglern?
		\item 	Wo kann Munition und Ausrüstung nachgefasst werden?
	\end{itemize}