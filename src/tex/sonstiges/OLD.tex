%TODO: Links, Erklärungen, Einleitung

\section{Das AGA-Handbuch}

\subsection{Über die Grundausbildung}

\subsection{Organisation der Trupps}


\subsubsection{Organisation und Befehlshierarchie}

%TODO: Erklärung warum und wie es funktioniert
	Das Ziel jeder militärischen Organisation ist es, die gesetzten Ziele schnell und effizient zu erreichen: wie kleine, fein abgestimmte Zahnräder sollen Befehlender und Befehlsgeber miteinander interagieren und auf rapide Veränderungen ebenso rapide reagieren. Dazu ist die Einhaltung der Befehlshierarchie, der Funk- und Meldediziplin absolut notwendig.\\
	Im englischsprachigen Raum existiert der Begriff \textit{chain-of-command}, welcher im deutschen Sprachraum äquivalent mit den Begriffen \textit{Dienstweg} und, besonders im militärischen, \textit{Befehlshierarchie} ist. Man versteht darunter, wie und an wen Befehle ausgegeben und angenommen werden. Ganz wichtig ist: Sofern nicht explizit anders geregelt (Grün: "Schwarz meldet Sichtungen direkt an \ac{OPL}") werden alle Befehle nur an direkt Unterstellte ausgegeben und Meldungen nur an direkte Vorgesetzte gemacht. Als Beispiel sei genannt, dass die Zugführung einem Soldaten im Regelfall nicht einen expliziten Befehl geben ("Sicherungsbereich Nord-nord-ost") wird, dafür ist der Truppführer zuständig. Dies geschieht, um eine klare Übersicht und Kontrolle auch in komplizierten und unübersichtlichen Situationen zu haben - und fußt auf Vertrauen. Dem Vertrauen, dass der Befehlende nur solche Befehle ausgibt, die der Befehlsempfänger unter der Berücksichtigung seiner momentanen Lage (Ausrüstung, Position, Zustand, ...) realistischer Weise schaffen kann. Umgekehrt vertraut ein Befehlender darauf, dass ein solcher Befehl ausgeführt und der Ausgang gemeldet wird: Kommt es beispielsweise zu momentan unauflösbaren Schwierigkeiten, werden diese gemeldet. \\
	Jeder sollte dabei im Hinterkopf behalten, dass...
		\begin{itemize}
			\item ... ein schlechter Befehl manchmal besser ist als kein Befehl.
			\item ... mit wenigen Ausnahmen jeder Spieler auch ein Mensch ist. 
			\item ... mit Herz und Verstand befehligt und gehorcht wird. 
			\item ... für jede Diskussion die \ac{AAR} da ist.
			\item ... das Ziel ist, jederzeit immer besser zu werden.
			\item ... eine "Scheißegal"-Attitüde dazu führen muss, dass du den Server verlässt.
		\end{itemize}

\subsubsection{Die Trupps}

%TODO: Ausführliche Beschreibung der Aufgabengebiete

\subsubsection{Uniformen und Zeichen}
	In den meisten Fällen lassen sich Mitglieder des \ac{TTT} an den Zeichen auf ihren Uniformen erkennen und identifizieren:
		\begin{itemize}
			\item Patch auf dem linken Oberarm in Truppfarbe und mit dem entsprechenden Logo
			\item Auf der Rückseite des Helms ist die Nummer des Soldaten im Trupp angebracht
			\item Auf jedem Handschuh befindet sich das Logo des \ac{TTT} in der entsprechenden Truppfarbe
		\end{itemize}

\subsection{Der Trupp}

Fasst man mindestens zwei Buddyteams zusammen, so entsteht ein klassischer Kommando-Trupp wie man ihn beispielsweise für den Häuserkampf oder Spezialeinsätze benötigt. Wie auch beim Buddyteam entsteht dabei die Stärke eines Trupps dadurch, dass sich die einzelnen Teams gegenseitig unterstützen und sichern können. Je besser ein Trupp aufeinander eingespielt ist, umso effizienter wird er. Und gerade bei einem 4-Mann-Trupp ist ein hohes Maß an Übung und Einspielen erforderlich, da bereits der Ausfall eines einzelnen Soldaten den Trupp signifikant an der Erfüllung seines Auftrags hindern kann. \\
	Ein Trupp aus drei Buddyteams ist nicht nur robuster, sondern auch taktisch vielfältiger einsetzbar: so kann beispielsweise ein Buddyteam unter der Sicherung der beiden anderen Teams vorrücken. Er ist der Standardtrupp, der im \ac{TTT} Verwendung findet. Wie in solcher Trupp aussehen kann ist nachfolgend aufgeführt:
		\begin{itemize}
			\item Buddy-Team 1, “Führung” 
			\begin{itemize}	
				\item Nr.1: Truppführer -  Führt den Trupp durch klare, vorbildliche Führung von vorn
				\item Nr.2: Stellv. Truppführer - Unterstützt den Truppführer indem er Funk oder ähnliche Aufgaben übernimmt
			\end{itemize}
		\end{itemize}

		\begin{itemize}
			\item Buddy-Team 2, “Schwere Waffen” 
			\begin{itemize}
				\item Nr.3: MG-Assistent - Trägt zusätzliche Munition für das MG, hilft beim einschießen auf die Ziele, lenkt das MG, trägt Munition, unterstützt mit Nahsicherung
				\item Nr.4: MG-Schütze - Trägt das MG und führt den Feuerkampf
			\end{itemize}
		\end{itemize}

		\begin{itemize}
			\item Buddy-Team 3, “Spezialisten” 
			\begin{itemize} 
				\item Nr.5: Spezialist 1 - Übernimmt zusammen mit seinem Buddy Spezialaufgaben wie beispielsweise Grenadier, \ac{AA}/\ac{AT}, \ac{EOD}, Sanitäter, Designated Marksman
				\item Nr.6: Spezialist 2 - Übernimmt zusammen mit seinem Buddy Spezialaufgaben wie beispielsweise Grenadier, \ac{AA}/\ac{AT}, \ac{EOD}, Sanitäter, Designated Marksman
			\end{itemize}
		\end{itemize}
Es ist nicht ungewöhnlich, dass man über seine Nummer im Trupp oder die Nummer des Buddyteams angesprochen wird. Es ist daher von Vorteil, sich mit Stift und Papier zu merken in welchem Buddyteam man ist und welche Nummer man selbst und die anderen Leute haben.