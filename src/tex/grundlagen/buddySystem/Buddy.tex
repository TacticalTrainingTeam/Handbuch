\newpage
\subsection{Dein Buddy und du}
\centerline{\textit{Dies ist mein Buddy. Davon gibt es viele, aber dieser gehört mir.}}
	Die Aussage einer Erzieherin \textit{"So, und jetzt nimmt jeder die Person links von ihm an die Hand"} fasst am besten zusammen, für was ein Buddyteam steht: Man passt gegenseitig aufeinander auf, sodass man nicht verloren geht. Egal ob beim Tauchen, bei einer Ralley, im Flugzeug, beim Pair-Programming oder im Scharfschützenteam - die Stärke der Kleinsten aller Gruppen besteht darin, dass zwei Personen an einem Strang ziehen und sich dabei gegenseitig unterstützen können. \\
	Das Buddyteam im \ac{TTT} ist viele Dinge: die kleinste organisatorische Einheit, das solideste Fundament jedes Trupps, die beste Versicherung eines einzelnen Spielers das Ende der Mission zu sehen. Buddyteams werden üblicherweise so gebildet, dass sie einer speziellen Funktion folgen können: Sei dies als Maschinengewehrcrew, Fahrzeugbekämpfung, Aufklärung, Führung oder jeder anderen denkbaren Aufgabe. Buddyteams haben auch Nachteile. Darunter ist beispielsweise, dass sie üblicherweise zufällig für eine Mission gebildet werden, die beiden Personen sich nicht notwendigerweise verstehen und ähnliches. Regelmäßiges Spielen und Trainingkann diese Probleme eindämmen, jedoch niemals ganz. \\
	Das zentrale Element eines solchen Teams bist immer \textit{DU} -- Dein Streben muss es ständig sein, ein besserer Buddy zu werden. Nach der AAR nicht nur Vorgesetzte und erfahrenere Spieler, sondern auch den für die Mission zugeteilten Buddy zu fragen was man verbessern kann sollte zum guten Ton eines \ac{TTT}-Spielers gehören. Auch während der Mission gibt es ein paar Dinge, über die man sich jederzeit im Klaren sein sollte:
		\begin{itemize}
			\item Position meines Buddys
			\item Zustand meines Buddys
			\item Ausrüstung, momentaner Munitionsstand meines Buddys
			\item Grobe Blickrichtung oder Sicherungsrichtung meines Buddys
			\item ...
		\end{itemize}
	Auch hier hilft nur eins: Erfahren, denken, üben. So hilft es beispielsweise nichts, einfach so loszustürmen um seinen Buddy zu bergen und dabei selbst schwer verwundet zu werden. 