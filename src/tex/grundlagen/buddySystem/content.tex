\section{Das Buddy-System}
	Das Buddy"=System ist ein grundlegender Bestandteil des \textbf{TTT}s (\ref{sec:Grundsaetze}). Dabei ist das Buddy"=Team die kleinste organisatorische Einheit, das solide Fundament jedes Trupps und die Lebensversicherung jedes Soldaten. Damit dieses System funktioniert ist es erforderlich, dass die Buddy"=Mitglieder aufeinander aufpassen, an einem Strang ziehen und sich gegenseitig unterstützen. Weiterhin können Buddy"=Teams als eine Einheit agieren, die spezielle Aufgaben übernehmen. Beispiele hierfür wären Schwerpunktwaffen, Fahrzeugbekämpfung, Aufklärung oder Führung, um nur einige Möglichkeiten zu nennen. Gegenüber den genannten Vorteilen existieren auch Nachteile, die nicht ungenannt bleiben sollen. Ein leicht erkenntlicher Nachteil ist, dass ein Buddy"=System aus Menschen besteht, die zusammenarbeiten müssen. Bis dieses Team optimal zusammenwirkt, kann unterschiedlich viel Zeit vergehen. Dies kann jedoch unter regelmäßiges anwenden beschleunigt werden. Weiterhin können die Spezialisierung von Buddy"=Teams in gewissen Situationen von Nachteil sein. Dies kann jedoch durch den Truppführer kompensiert werden. Damit das Buddy"=System alle Qualitäten hervorbringt sollte jeder Spieler, sowohl Stammspieler, als auch Gast, danach streben ein besserer Buddy zu werden.
	
	Während der Mission sollte sich ein Spieler über folgenden Dinge seines Buddys im Klaren sein:
	\begin{itemize}
		\item Position
		\item Zustand 
		\item Ausrüstung und momentaner Munitionsstand
		\item Grobe Blickrichtung oder Sicherungsrichtung
	\end{itemize}
	Diese Informationen sollten durch die Kommunikation mit seinen Buddy aktuelle gehalten werden (kleine Kampfgespräch). Beispiele hierfür sind z.\,B. die Beobachtungsrichtung, Feindbekämpfung, Positionswechsel oder Richtungsänderungen. Im  \ac{CQB} erfolgt die Kommunikation in anderer Form (siehe \ref{CQB}).
	Nicht zu vergessen ist, dass Eigensicherung vor Fremdsicherung geht. Das heißt, der Buddy wird erst dann gerettet, wenn sichergestellt ist, dass die Rettung erfolgreich verläuft. Denn ein verletzter Buddy kann seinen Buddy nicht effektiv helfen.