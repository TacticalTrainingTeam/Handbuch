\newpage
\subsection{Die Befehle – die wichtigsten Kommandos im TTT}

	Die häufigsten Kommandos eines Trupp- oder Zugführers \\

\subsubsection{Kommando lautet <<Deckung!>>}

	Die Deckung ist ein situationsbedingter Aufenthaltsort \\ 

	Wird man vom Feind überrascht, ist der Soldat angewiesen, sofort Deckung im nahen Umfeld (Radius 20 Meter) zu suchen – ohne die Trupp-Struktur aufzulösen. Dabei gilt zu beachten, dass die Buddy-Teams immer im Verbund bleiben. \\

	Als Deckung zählt alles, was als beschusssicher gilt: Häuser, Felsen, Mauern, Senken. Ist keine Deckung in Laufweite, wird sofort Bodenlage eingenommen und auf den Feind ausgerichtet. Bei indirektem Beschuss (Granaten, etc.) wird ein besonders weiter Abstand zwischen den Kameraden gesucht.\\

	Der Befehl vom Truppführer lautet <<Deckung!>> oder als ausführlicheres Beispiel <<Deckung, 6 Uhr, hinter der Mauer!>> \\

\subsubsection{Kommando: <<Bekämpfen …!>>}

	Der Soldat soll einen erkannten Feind bekämpfen. Hierbei ist es wichtig, genau abzuwägen, wie viel Einweisung benötigt wird, damit das Kommando erfolgreich ausgeführt werden kann. Der Soldat meldet bei erfolgreicher Absolvierung des Auftrags <<Gegner, YXZ, bekämpft!>> \\

\subsubsection{Kommando: <<Feuer frei !>>}

	Das ist die direkte Anweisung, JETZT zu schießen. Kann ergänzt werden mit <<Feuern, 3 Salven>> oder sonstigen Einschränkungen. \\

\subsubsection{Kommando: <<Unterdrückungsfeuer!>>}

	Das ist die direkte Anweisung, in die generelle Richtung des Gegners zu schießen, um ihn in die Deckung zu zwingen und ihn an Gegenfeuer oder Manövern zu hindern. \\

\subsubsection{Kommando <<Gezieltes Feuer!>>}

	Das ist die direkte Anweisung, den Gegner mit gezielten Salven oder Einzelschüssen direkt zu treffen – also das Gegenteil vom Unterdrückungsfeuer. In der Regel wird bei der Feuerfreigabe erwartet, dass gezieltes Feuer abgegeben wird. \\

\subsubsection{Kommando <<Feuerfreigabe erteilt>>}
	Grundsätzliche Feuerfreigabe. Das heißt, sobald Gegner in Sicht und der Feuerkampf ist aussichtsreich, darf geschossen werden. Gilt bis zum Widerruf. \\

\subsubsection{Kommando <<Feuer nur auf Freigabe>>}
	Vor dem Schuss muss Feuerfreigabe angefordert werden. \\

\subsubsection{Kommando: <<Stopfen!>>}
	Der Soldat stellt sofort das <<Schießen>> ein. \\

\subsubsection{Kommando lautet <<Tarnen!>>}

	Das bedeutet, dass man sich in den Schutz von Büschen, Bäumen oder Sonstigem begeben soll. Wichtig beim Tarnen ist, dass man vom Gegner nicht erkannt wird. Dabei gilt es die Grundregeln zu beachten (Sky-lineing, etc.). Gras taugt nicht als Tarnung! \\

\subsubsection{Kommando: <<Bezieht Stellung bei …!>>}

	Die Stellung ist ein selbst gewählter Aufenthaltsort. \\

	Die Stellung ist idealerweise ein gewählter und geschützter Aufenthaltsort (Haus, Senke, Felsen, Mauer) und lässt sich leicht verteidigen. Die Soldaten haben dabei maximale Deckung eingenommen und sichern. \\

	Der Truppführer gibt den Befehl aus, um einen Aufenthaltsort festzulegen, wo sich der Trupp aufhalten soll. Daraus folgt automatisch, dass der Trupp den Ort erreicht und vorläufig sichert. Der Sicherungsschwerpunkt liegt automatisch auf vermutete Feindstellungen. Mit dem Zusatz <<gedeckt>> betont er, dass der 		Trupp eine Aufklärung durch den Feind unter allen Umständen vermeiden soll. In der Praxis: Nähern in die gedeckte Stellung unter Sicht- und Deckungsschutz. Dort in die Stellung getarnt beziehen und halten. In der Stellung erfolgt vom Truppführer:
		\begin{itemize}
			\item Kontaktaufnahme mit dem Zugführer 
			\item Geländetaufe 
			\item Verteilung von Feuerbereichen 
			\item Vorausschauende Planung (Wegplanung, Korridore, weitere Stellungen) 
			\item Kommando: <<Bezieht gedeckte Stellung bei …!>> 
		\end{itemize}

\subsubsection{Kommando: <<Wechselt die Stellung!>> (Nein, nicht was ihr denkt …)}

	Der taktische Stellungswechsel ist ein eigenständiger Positionswechsel in eine andere Stellung – beispielsweise, weil man befürchtet, die Stellung ist vom Feind bereits aufgeklärt wurde und damit Feindfeuer erwartet. Das gilt besonders nach Feuerüberfällen – häufige Stellungswechsel sind sehr effektiv um Gegner zu verwirren. \\

\subsubsection{Kommando:<<Nehmt ein!>>}

	Das bedeutet für den Trupp, dass eine gegnerische Stellung und/oder eines Gebiet (umkämpft oder nicht umkämpft) unter allen Umständen eingenommen und erobert werden soll. Einnehmen signalisiert dem Soldaten, dass Feindkontakt möglich ist. \\

\subsubsection{Kommando: <<Haltet …!>>}

	Das bedeutet nichts anderes – als „bleiben Sie IN der Stellung und verteidigen Sie die Stellung>>. \\

\subsubsection{Kommando: <<Melden bei Vollzug!>>}

	Der Soldat soll, sobald er seinen Auftrag erledigt hat, sich beim Auftraggeber melden – per Funk oder direkter Ansprache. \\

\subsubsection{Kommando: <<Status>>}

	Der Soldat gibt den Gesundheitsstatus und seinen Munitionsstatus, mittels Ampelsystem, durch. (Beispiel: << Hier 5, Grün, Gelb >>) Ggf. können weitere Meldungen kurz und knackig mit gegeben werden. (Beispiel: <<noch 500 Meter bis zu eurer Stellung >>) \\

\subsubsection{Feuerfreigaben}

	Es versteht sich von selbst, dass wir nicht auf alles zur jeder Zeit schießen können. Daher gibt es die sogenannten Feuerfreigaben, wann geschossen werden darf und wann nicht. Wir unterschieden 3 Typen: \\

\paragraph{<<Status - Feuer halten>>  oder auch <<Feuerstatus rot>> }

	Nur schießen, wenn es um das blanke Überleben geht. „Feuer halten“ wird ausgegeben, wenn wir leise und unerkannt in eine Stellung einsickern wollen. Feuergefechte sollten unter allen Umständen vermieden werden, da Schüsse die eigene Position verraten und Kameraden gefährden können. \\

\paragraph{<<Status - Feuer erwidern>>  oder auch <<Feuerstatus gelb>> }

	Bei erkanntem Feind, der uns noch nicht uns bekämpft (kalt), wird der Feind gemeldet und es wird gewartet, was der Truppführer entscheidet. Es darf bei Beschuss auf die eigene Stellung zurückgeschossen werden, wenn der Feind klar erkannt wurde (heiß). \\

\paragraph{<<Status - Feuer frei>> oder auch <<Feuerstatus grün>> }

	Jeder erkannte Feind darf bekämpft werden. Diese Freigabe wird beispielsweise bei einem Feuerüberfall gegeben, oder in einem Verteidigungsszenario aus einer befestigten Stellung heraus. \\