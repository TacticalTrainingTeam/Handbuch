\subsection{Der Befehle}
\begin{quote}
	\glqq Richtig ist besser\grqq
\end{quote}
Befehle sind laut Lehrbuch eine „Anweisung zu einem bestimmten Verhalten, die ein Vorgesetzter einem Untergebenen mit dem Anspruch auch Gehorsam erteilt“. Bei uns einfach „Mach dies und das und möglichst zum Zeitpunkt X“. Der Befehl ist in der Regel ein Gespräch bei uns zwischen Befehlsgeber und Befehlsempfänger. Der Befehl enthält einen Auftrag.\\
Der Auftrag ist das Kernstück und enthält die Informationen, auf die es ankommt.

\paragraph*{Befehle sind verbindlich!}\ \\
Befehle des „Vorgesetzten“ sind während der Mission umzusetzen, außer sie basieren auf fataler Fehleinschätzung aufgrund mangelnder Informationen, die dem Vorgesetzten nicht vorliegen. Dann gilt: Vorgesetzen informieren (beispielsweise über Feindkräfte, Lage, mangelnder Ausrüstung) und neue Befehlsausgabe abwarten. Absolut zu vermeiden sind Diskussionen. Dafür haben wir eine Nachbesprechung, wo alle Fehler und Fehlentscheidungen besprochen werden können. Weiterhin hat eine schlichte Befehlsverweigerung gewisse Konsequenzen.\\

\paragraph*{Befehle sind kurz und präzise!}\ \\
Richtig Befehle auszugeben ist für uns Spieler nicht einfach. 
Neben den vielen „öhm, ehm“ kommt besonders über Funk gerne mal kompletter Unsinn aus dem eigenen Mund. 
Weil wir erst während dem sprechen unsere Gedanken sortieren. 
Daher gilt „erste denken, dann sprechen“ und für Funksprüche „erst denken, drücken, sprechen“. Lieber ein paar Sekunden warten, konzentrieren, sortieren und dann die Befehlsausgabe.
Im folgenden bekommt wird der Aufbau des Befehls erläutert. Regel Nr.1: Konzentriert euch auf die Information, lasst den Schnickes und blümerante Dinge weg.
Faustregel zur Länge: Bleibt kurz und präzise. Floskeln weglassen – und lieber später mal den Kameraden für die gute Umsetzung loben. 

\paragraph*{Möglicher Aufbau eines Befehls}
\begin{enumerate}
	\item Vorwarnung
	\item Rückmeldung
	\item Kurze Beschreibung der Situation (optional)
	\item Auftrag
	\item Auftragsbestätigung
	\item Auftragsauslösung
\end{enumerate}

Beispiel eines Befehls:
\begin{enumerate}
	\item ZF: "Trupp Rot, bereit machen, neuer Auftrag in 1 Minute!"
	\item TF: "Trupp Rot bereit!"
	\item ZF: "Trupp Rot, aufgepasst: Lage und Vorhaben: Wir müssen im Dorf Kabalis eine Stellung errichten, dazu wollen wir ein Gebäude einnehmen. Trupp Blau wird den Vorstoß von der jetzigen Stellung hier absichern. Trupp Rot wird das Haus einnehmen. Wir erwarten nur leichten Widerstand -- evtl. ein paar verstreute Infanteristen in der Umgebung. So verstanden?"
	\item TF: "So verstanden!"
	\item ZF: "Trupp Rot, Euer Auftrag: Einnehmen des weißen, flachen Gebäudes mit rotem Dach auf 320 Grad, 200 Meter entfernt! Verstanden?"
	\item TF: "Verstanden, Einnehmen des weißen, flachen Gebäudes mit rotem Dach auf 320 Grad, 200 Meter entfernt."
	\item ZF: „Gut, ausführen wenn bereit!“ (Truppführer entscheidet Zeitpunkt)\\ 
		ZF: „Gut, sofort ausführen!/ Ausführen bei Erreichen des Punktes X, Ausführen bei 0815 Ortszeit“ (Zugführer entscheidet, wann und wo).
	\item ZF: "Meldung bei Vollzug!"
\end{enumerate}

\subsubsection{Der Auftrag}
Beispielhafter Aufbau eines Auftrags:\\

In oben genanntem Beispiel ist folgender Auftrag zu finden:
"Trupp Rot, Euer Auftrag: Einnehmen des weißen, flachen Gebäudes mit rotem Dach auf 320 Grad, 200 Meter entfernt! Verstanden?"

Der Auftrag kommt mit dem Kommando „Einnehmen“ (siehe \ref{sec:kommando}). Nach dem Kommando kommt das Ziel, welches eingenommen werden soll. Danach noch Richtung und Entfernung, damit der Auftragsempfänger auch das einnimmt, was der Auftragsgeber meint. Das Ganze übersichtlich dargestellt:

\begin{enumerate}
	\item Kommando
	\item Ziel
	\item Ortsangabe (Richtung \& Entfernung)
\end{enumerate}

\paragraph*{Auftragstaktik}\ \\
\begin{quote}
	\glqq Auftragstaktik Baby! Im \ac{TTT} arbeiten wir mit Aufträgen -- nicht mit der Hundeleine.\grqq
\end{quote}
Gute „Führer“ arbeiten IMMER auftragsbasiert -- also eine dezentrale Befehlsausgabe mit Vertrauen, dass der Auftragsempfänger den Auftrag selbständig im Sinne des Befehls erledigt. Man gibt als einen Auftrag – aber NICHT wie dieser Auftrag erreicht wird. Es kann dem Zugführer in der Regel völlig egal sein, über welchen Eingang das Haus genommen wird. Auftrag erteilen und alles Weitere dem Truppführer überlassen, er kann das am besten entscheiden, wie er seine Leute sicher und erfolgreich ins Haus bekommt.\\

Das Gleiche gilt für Truppführer: Hört auf mit dem Mikromanagement! Wenn ihr Eurem Trupp das Kommando gebt „Deckung“ dann muss nicht gesagt werden, hinter welchen Felsen das jetzt sein soll. Dass gleiche gilt für „Stellung“ – der Trupp wird automatisch Stellung beziehen und bei Erfolg den Vollzug durchgeben. Wenn ein Truppführer einen Auftrag erteilt, beispielsweise ein Kontakt zu bekämpfen, dann geht das folgendermaßen:\\
\begin{tabular}{ll}
	& TF: 5-6, 2 Feindkräfte auf 120 Grad, 200 Meter, am Spitzbaum, gesichtet?\\
	& 5-6: „Gesichtet!“\\
	& TF: 5-6, Stellungswechsel, dann bekämpfen, Meldung bei Vollzug!“\\
	& TF: „Feuer frei!“ (Feuerfreigabe)\\
\end{tabular}


5 und 6 haben ihren Auftrag: Sie suchen sich eine geeignete Stellung mit genügend Deckung, überprüfen kurz die Flanken (rechts und links), sprechen sich kurz ab, legen Feuermodus fest und feuern nach Absprache untereinander. Danach melden Sie dem TF per Funk das erfolgreiche bekämpfen des Gegners. Anschließend verlegen sie zurück in Ausgangsstellung. 
  
\subsubsection{Kommandos -- die wichtigsten Befehle}
\label{sec:kommando}
\paragraph*{Kommando: <<Deckung!>>}\ \\
	Die Deckung ist ein situationsbedingter Aufenthaltsort.\\ 
	Wird man vom Feind überrascht, ist der Soldat angewiesen, sofort Deckung im nahen Umfeld (Radius 20 Meter) zu suchen – ohne die Trupp-Struktur aufzulösen. Dabei gilt zu beachten, dass die Buddy-Teams immer im Verbund bleiben. \\
	Als Deckung zählt alles, was als beschusssicher gilt: Häuser, Felsen, Mauern, Senken. Ist keine Deckung in Laufweite, wird sofort Bodenlage eingenommen und auf den Feind ausgerichtet. Bei indirektem Beschuss (Granaten, etc.) wird ein besonders weiter Abstand zwischen den Kameraden gesucht.\\

	Der Befehl vom Truppführer lautet <<Deckung!>> oder als ausführlicheres Beispiel <<Deckung, 6 Uhr, hinter der Mauer!>>

\paragraph*{Kommando: <<Bekämpfen \dots!>>}\ \\
	Der Soldat soll einen erkannten Feind bekämpfen. Hierbei ist es wichtig, genau abzuwägen, wie viel Einweisung benötigt wird, damit das Kommando erfolgreich ausgeführt werden kann. Der Soldat meldet bei erfolgreicher Absolvierung des Auftrags <<Gegner, YXZ, bekämpft!>>

\paragraph*{Kommando: <<Feuer frei!>>}\ \\
	Das ist die direkte Anweisung, JETZT zu schießen. Kann ergänzt werden mit <<Feuern, 3 Salven>> oder sonstigen Einschränkungen.

\paragraph*{Kommando: <<Unterdrückungsfeuer!>>}\ \\
	Das ist die direkte Anweisung, in die generelle Richtung des Gegners zu schießen, um ihn in die Deckung zu zwingen und ihn an Gegenfeuer oder Manövern zu hindern.

\paragraph*{Kommando <<Gezieltes Feuer!>>}\ \\
	Das ist die direkte Anweisung, den Gegner mit gezielten Salven oder Einzelschüssen direkt zu treffen -- also das Gegenteil vom Unterdrückungsfeuer. In der Regel wird bei der Feuerfreigabe erwartet, dass gezieltes Feuer abgegeben wird.

\paragraph*{Kommando <<Feuerfreigabe erteilt>>}\ \\
	Grundsätzliche Feuerfreigabe. Das heißt, sobald Gegner in Sicht und der Feuerkampf ist aussichtsreich, darf geschossen werden. Gilt bis zum Widerruf.

\paragraph*{Kommando <<Feuer nur auf Freigabe>>}\ \\
	Vor dem Schuss muss Feuerfreigabe angefordert werden.

\paragraph*{Kommando: <<Stopfen!>>}\ \\
	Der Soldat stellt sofort das <<Schießen>> ein.

\paragraph*{Kommando: <<Bezieht Stellung bei \dots!>>}\ \\
	Die Stellung ist ein selbst gewählter Aufenthaltsort. \\
	Die Stellung ist idealerweise ein gewählter und geschützter Aufenthaltsort (Haus, Senke, Felsen, Mauer) und lässt sich leicht verteidigen. Die Soldaten haben dabei maximale Deckung eingenommen und sichern. \\
	Der Truppführer gibt den Befehl aus, um einen Aufenthaltsort festzulegen, wo sich der Trupp aufhalten soll. Daraus folgt automatisch, dass der Trupp den Ort erreicht und vorläufig sichert. Der Sicherungsschwerpunkt liegt automatisch auf vermutete Feindstellungen. Mit dem Zusatz <<gedeckt>> betont er, dass der Trupp eine Aufklärung durch den Feind unter allen Umständen vermeiden soll. In der Praxis: Nähern in die gedeckte Stellung unter Sicht- und Deckungsschutz. Dort in die Stellung getarnt beziehen und halten. In der Stellung erfolgt vom Truppführer:
		\begin{itemize}
			\item Kontaktaufnahme mit dem Zugführer 
			\item Geländetaufe 
			\item Verteilung von Feuerbereichen 
			\item Vorausschauende Planung (Wegplanung, Korridore, weitere Stellungen) 
			\item Kommando: <<Bezieht gedeckte Stellung bei \dots!>> 
		\end{itemize}

\paragraph*{Kommando: <<Wechselt die Stellung!>>}\ \\
	Der taktische Stellungswechsel ist ein eigenständiger Positionswechsel in eine andere Stellung – beispielsweise, weil man befürchtet, die Stellung ist vom Feind bereits aufgeklärt wurde und damit Feindfeuer erwartet. Das gilt besonders nach Feuerüberfällen – häufige Stellungswechsel sind sehr effektiv um Gegner zu verwirren.

\paragraph*{Kommando: <<Nehmt \dots\ ein!>>}\ \\
	Das bedeutet für den Trupp, dass eine gegnerische Stellung und/oder eines Gebiet (umkämpft oder nicht umkämpft) unter allen Umständen eingenommen und erobert werden soll. Einnehmen signalisiert dem Soldaten, dass Feindkontakt möglich ist.

\paragraph*{Kommando: <<Haltet \dots!>>}\ \\
	Das bedeutet nichts anderes – als „bleiben Sie IN der Stellung und verteidigen Sie die Stellung>>.

\paragraph*{Kommando: <<Melden bei Vollzug!>>}\ \\
	Der Soldat soll, sobald er seinen Auftrag erledigt hat, sich beim Auftraggeber melden – per Funk oder direkter Ansprache.

\paragraph*{Kommando: <<Status>>}\ \\
	Der Soldat gibt den Gesundheitsstatus und seinen Munitionsstatus, mittels Ampelsystem, durch. (Beispiel: << Hier 5, Grün, Gelb >>) Ggf. können weitere Meldungen kurz und knackig mit gegeben werden. (Beispiel: <<noch 500 Meter bis zu eurer Stellung >>)

\subsubsection{Feuerfreigaben}
	Es versteht sich von selbst, dass wir nicht auf alles zur jeder Zeit schießen können. Daher gibt es die sogenannten Feuerfreigaben, wann geschossen werden darf und wann nicht. Wir unterschieden 3 Typen:

\paragraph*{<<Feuerstatus rot>>}\ \\
	Alternativ: <<Status -- Feuer halten>>\hfil\\
	Nur schießen, wenn es um das blanke Überleben geht. <<Feuer halten>> wird ausgegeben, wenn wir leise und unerkannt in eine Stellung einsickern wollen. Feuergefechte sollten unter allen Umständen vermieden werden, da Schüsse die eigene Position verraten und Kameraden gefährden können.

\paragraph*{<<Feuerstatus gelb>>}\ \\
	Alternativ: <<Status -- Feuer erwidern>>\hfil\\
	Bei erkanntem Feind, der uns noch nicht uns bekämpft (kalt), wird der Feind gemeldet und es wird gewartet, was der Truppführer entscheidet. Es darf bei Beschuss auf die eigene Stellung zurückgeschossen werden, wenn der Feind klar erkannt wurde (heiß).

\paragraph*{<<Feuerstatus grün>>}\ \\
	Alternativ: <<Status -- Feuer frei>>\hfil\\
	Jeder erkannte Feind darf bekämpft werden. Diese Freigabe wird beispielsweise bei einem Feuerüberfall gegeben, oder in einem Verteidigungsszenario aus einer befestigten Stellung heraus.

\pagebreak	
\subsubsection{Zusammenfassung Kommandos}
	\begin{longtable}{p{0.3\linewidth}p{0.65\linewidth}} 
		\toprule
		\textbf{Befehl} & \textbf{Bedeutung}\\
		\midrule
		Deckung! & Beschusssichere Deckung in 20\,m Umkreis suchen mit Buddy, ansonsten auf den Boden legen.\\
		Bekämpfen! & Erkannten Feind bekämpfen, bei Vollzug mit \glqq Bekämpft\grqq\, bestätigen.\\
		Feuer Frei! & Feuer sofort eröffnen.\\
		Unterdrückungsfeuer! & Feuer in Feindrichtung, um diesen am kämpfen oder bewegen zu hindern.\\
		Gezieltes Feuer! & Mit gezieltem Feuer den Feind bekämpfen. Wird bei \glqq Feuer frei\grqq\, vorausgesetzt.\\
		Feuerfreigabe erteilt! & Ist Feuerkampf aussichtsreich, darf dieser begonnen werden.\\
		Feuer nur auf Freigabe! & Feuerkampf ist nur auf expliziten Befehl hin zu beginnen.\\
		Stopfen! & Das Feuern ist \textbf{sofort} einzustellen.\\
		Bezieht Stellung bei \dots ! & Mit Trupp oder Buddy zu einer Position gehen und entsprechende Sicherung aufbauen.\\
		Wechselt die Stellung! & Mit Trupp oder Buddy oder alleine in andere Stellung verlegen. Auch: \glqq Stellungswechsel nach \dots\grqq\\
		Nehmt \dots ein! & Das angegebene Ziel ist unter allen Umständen zu erobern und zu sichern.\\
		Haltet! & In der Stellung bleiben und diese verteidigen.\\
		Melden bei Vollzug! & Über den Ausgang des (vorangegangenen) Befehls informieren. \\
		Status! & Gesundheitsstatus und Munitionsstatus mit Ampelsystem durchgeben.\\
		\bottomrule
	\end{longtable}
