\newpage
\subsection{Artillerie}
\subsubsection{Allgemeines}
\paragraph{Begriffserklärung}
	Typen
	\begin{longtable}{p{0.4\linewidth} p{0.6\linewidth} }
		Typ I 		& 		Feuerfreigabe, ohne Schussverzögerung \\
		Typ II 		&		Feuerfreigabe nur auf Befehl \\
	\end{longtable}

	Arten
	\begin{longtable}{p{0.4\linewidth} p{0.6\linewidth} }
		Flächenangriff 	&	vom Zielpunkt aus in einem vorgegebenen Radius \\
		Linienangriff 		&	vom Ziel aus in eine vorgegebene Himmelsrichtung \\
		Punktangriff 		&	Beschuss auf das Ziel \\
	\end{longtable}

	Sonstige Begriffe

	\begin{longtable}{p{0.4\linewidth} p{0.6\linewidth} }
		FDC 		&		Fire Direction Commander \\
		CI  		&		CheckIn \\
		FDCB 		&		Fire Direction Command Briefing (Feuerbefehl) \\
		FO  		&		Forward Observer \\
		ETA  		&		Zeit vom Abschuss bis Einschlag \\
		Continue 	&		Feuerbefehl fortfahren \\
		Standby 	&		Feuerbefehl pausieren \\
		Break  	&		Feuerbefehl abbrechen \\
		Shot over  	&		Meldet das erster Schuss abgegeben wurde \\
		Shot out  	&		Meldet das letzter Schuss abgegeben wurde \\
		Splash  	&		signalisiert Einschlag, wird 5sek vor Einschlag gemeldet \\
		Offset  	&		beschreibt eine Verschiebung vom Ausgangsziel ausgehend \\
	\end{longtable}

\paragraph{Arbeitsgeräte}
	Mk6 Mörser
	\begin{longtable}{p{0.4\linewidth} p{0.6\linewidth} }
		- Geschossart / -anzahl  	& - 18x, 82mm HE \\
		 				& - 8x Rauchgranaten (weiß) \\
		 				& - 8x Leuchtgeschosse (weiß) \\
		- Kampfreichweite/  	& - min. Entfernung 30m \\
		Wirkung  			& - max. Entfernung 4000m \\
		 				& - ein HE-Geschoss hat einen Sprengradius \\
		 				& von ~20m \\
	\end{longtable}

	M4 Scorcher / M109A6
	\begin{longtable}{p{0.4\linewidth} p{0.6\linewidth} }
		Geschossart / -anzahl  	& - 32x, 155mm HE \\
						& - 6x Rauchgranaten (weiß) \\
						& - 2x Gelenkt \\
 						& - 2x Laser Gelenkt \\
 						& - 2x Cluster HE \\
 						& - 6x Cluster Minen (AT) \\
 						& - 6x Cluster Minen (AP) \\
		- Kampfreichweite /  	& - min. Entfernung 800m \\
		Wirkung  			& - max. Entfernung 68km \\
 						& - ein HE-Geschoss hat einen Sprengradius von ~40m \\
	\end{longtable}

	M5 Sandstorm MRLS
	\begin{longtable}{p{0.4\linewidth} p{0.6\linewidth} }
		- Geschossart / -anzahl  	& - 12x, 230mm AG Titanraketen \\
		- Kampfreichweite / 	& - min. Entfernung 800m \\
		Wirkung  			& - max. Entfernung 75km \\
						& - eine Rakete hat einen Sprengradius von ~80m \\
	\end{longtable}

\subsubsection{Mörser}

\subsubsection{Beispiele}
	Beispiel – Feuerbefehl (Typ II) \\

	FDCB: \\

	 Grün:>>Hammer, hier Grün, bereit für Feuerbefehl? Kommen.<< \\
	 Hammer:>>Hier Hammer, bereit für Feuerbefehl. Kommen.<< \\
	 Grün:>>Feuerbefehl lautet,  \\
	 - Typ II \\
	 - Flächenangriff 20, \\
	 - auf 1988 -(>>Strich<<) 0567, \\
	 - 37, \\
	 - 8, \\
	 - HE, \\
	 - alle 5 Sekunden, melden wenn Feuerbereit, \\
	 kommen.<< \\


	Hammer:   \\

	 >>Wiederhole  \\
	 - Typ II  \\
	 - Flächenangriff 20,  \\
	 - auf 1988 -(>>Strich<<) 0567,  \\
	 - HE,  \\
	 - alle 5 Sekunden, melden wenn Feuerbereit,  \\
	 kommen.<<  \\
	 Grün:>>Bestätige, Grün ende.<<  \\
	 Feuerbereitschaft:  \\
	 Hammer:>>Grün, hier Hammer, melde Feuerbereitschaft, ETA 21, kommen.<<  \\
	 Grün:>>Hier Grün, so verstanden, Continue.<<  \\
	 Hammer:>>Shot over.<<  \\
	 (Hammer:>>Splash.<<)  \\
	 Hammer:>>Shot out.<<  \\


	Reattack:  \\

	 Grün:>>Hammer, hier Grün, bereit machen für Reattack, kommen.<<  \\
	 Hammer:>>Hier Hammer, bereit für Reattack, kommen.<< \\
	 Grün:>>Offset Nord 20 Süd 10, kommen.<< \\
	 Hammer:>>Wiederhole Nord 20 Süd 10, kommen.<< \\
	 Grün:>>Bestätige, Grün ende.<< \\
	 Hammer:>>Grün, hier Hammer, melde Feuerbereitschaft, ETA 22, kommen.<< \\
	 Grün:>>Hier Grün, so verstanden, Continue.<< \\
	 Hammer:>>Shot over.<< \\
	 (Hammer:>>Splash.<<) \\
	 Hammer:>>Shot out.<< \\

	Feuerbefehl beenden: \\

	 Grün:>>Hammer, hier Grün, melde Feuerbefehl beendet, kommen.<< \\
	 Hammer:>>Hier Hammer, so bestätigt, Feuerbefehl beendet, kommen.<< \\
	 Grün:>>Gute Arbeit Hammer, Grün ende.<< \\



\subsubsection{Artillerie anfordern (5-Liner)}