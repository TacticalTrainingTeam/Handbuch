\pagebreak

\subsection{Markierung von Feindeinheiten}
\label{EnemyUnits}

Die Feindeinheiten werden immer mit >>E<< bezeichnet. Wichtig ist zu Kennzeichnen wann die Einheiten aufgeklärt wurden und diese entsprechend auf der Karte zu aktualisieren. \\
Die Markierung erfolgt nach dem Schema Art, Anzahl, Bewegung, Status, Uhrzeit. Beispiel: EI 5 St  C, 1230

\begin{longtable}{|p{0.3\linewidth}|p{0.05\linewidth}|p{0.35\linewidth}|p{0.2\linewidth}|}
	\hline
	Oberbegriff		&	Abk.	&	Anmerkung 	&	Beispiel \\ 
	\hline
	Enemy Infantery (Feindliche Infanterie)		&	EI	&	Feindliche Infanterie erkannt und aufgeklärt	&	\\ 
	\hline
	Enemy Tank (Feindlicher Kampfpanzer)	&	ET	&	Feindliches Panzerfahrzeug erkannt und aufgeklärt	&	\\ 
	\hline	
	Enemy Air (Feindliches Flugzeug)	&	EA	&	Feindliche Jets	&	\\ 
	\hline	
	Enemy (Armoured) Personal Carrier (Feindlicher Truppentransporter)	&	EPC	&	Einheiten wie BTR, BMP - also feindliche, gepanzerte Truppentransporter	&	\\ 
	\hline	
	Enemy Helicopter	&	EH	&	Feindliche Helikopter	&	\\ 
	\hline	
	Enemy Anti Air	&	EAA		&	Feindliche Anti-Air-Einheit	&	\\ 
	\hline	
	Static	&	St	&	statische Einheit	&	\\ 
	\hline	
	Patroullie	&	&	Einheit läuft ein definierten Weg in berechenbaren Muster	&	\\ 
	\hline	
	Mobil	&	Mb	&	Einheit ist mobil, Bewegungsmuster nicht erkennbar &	\\ 
	\hline	
	Patroullie	&	Pt	&	Einheit läuft ein definierten Weg in berechenbaren Muster	&	\\ 
	\hline
	Heiß	&	H	&	Einheit hat uns aufgeklärt, sucht aktiv nach uns	&	\\ 
	\hline	
	Kalt	&	C	&	Einheit hat uns noch nicht aufgeklärt, verhällt sich passiv	&	\\ 
	\hline		
\end{longtable}