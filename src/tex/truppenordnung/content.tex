\chapter{Truppenordnung}
In diesem Abschnitt werden die einzelnen Strukturelemente und die Organisation des \ac{TTT} während einer typischen Mission erklärt.\\
Die hier vorgestellten Strukturen basieren dabei auf der im Laufe der Zeit gesammelten Erfahrung, was in ArmA funktioniert und was nicht und sind verpflichtende Vorgabe für jeden Missionsbauer. Ausnahmen von diesen Strukturen müssen vorher angefragt und genehmigt werden.\\
Da im \ac{TTT} sowohl deutsche als auch amerikanische Strukturen benutzt werden, werden die amerikanischen Namen jeweils in Klammern zusätzlich zu den deutschen Namen genannt.

\section{\acf{OPL} / \acf{HQ}}
\begin{wrapfigure}{r}{0.35\textwidth}
	%\vspace{-15pt}
	\centering 
	\includegraphics[width=0.3\textwidth]{../img/truppenordnung/opl/opl}
	%\caption{Beispiel einer \ac{OPL}}
	%\vspace{-30pt}
\end{wrapfigure}	

Die \ac{OPL} ist die höchste Instanz innerhalb einer Mission. Sie hat den Oberbefehl über alle Einheiten inne, koordiniert das allgemeine Vorgehen innerhalb der Mission und verwaltet die Zuordnung der unterstützenden Einheiten zu den kämpfenden Einheiten. Sie kommuniziert grundsätzlich nur über die Long-Range mit ihren untergeordneten Einheiten.
\par\bigskip
Die \ac{OPL} ist folgendermaßen aufgebaut:
\begin{itemize}
	\item Operationsleiter\,/\,\acs{OPL} (\acf{CO}): Der Oberbefehlshaber der Mission. Er gibt die Befehle und erstellt "den großen Plan".
	\item stellv. \ac{OPL} (\acf{XO}): Unterstützt den \ac{OPL} bei seinen Aufgaben, typischerweise beim Funken mit den untergeordneten Trupps. Kann jedoch auch alle weiteren Aufgaben übernehmen, die ihm der \ac{OPL} überträgt -- er ist Mädchen für alles. Bewährt hat sich das Prinzip, dass der \ac{OPL} den eingehenden Funk übernimmt (Anfragen von anderen Trupps) und der stellv. \ac{OPL} den ausgehenden Funk (Abfragen von Statusberichten, Übermittlung von neuen Befehlen).
\end{itemize}
Ergänzt werden kann die \ac{OPL} durch maximal 4 Spieler, welche folgende Rollen einnehmen können:
\begin{itemize}
	\item Funker (\acf{RO}): ein zusätzlicher Funker, um die \ac{OPL} zu unterstützen, kann auf eine Spezialrolle beschränkt sein und für diese eine eigene LR-Frequenz bekommen (so kann es z.\,B. bei einer Mission mit vielen Lufteinheiten sinnvoll sein, jemanden zu haben, der sich auf einer eigenen Frequenz nur um die Koordination der Lufteinheiten um das Flugfeld kümmert und zentraler Ansprechpartner aller Lufteinheiten für Start-\,/\,Landemanöver ist)
	\item Aufklärungsoffizier (\acf{IO}): Sammelt alle verfügbaren, relevanten Daten und leitet diese gegebenenfalls an andere Trupps weiter. Hat meistens eine eigene, "große" Drohne (Greyhawk\,/\,Global Hawk) zur Feindaufklärung.
	\item freie Rolle (maximal einmal): je nach Mission kann es sinnvoll sein, dem \ac{OPL} einen Sanitäter, einen Nahsicherer, einen Fahrer o.\,Ä. zur Seite zu stellen
\end{itemize}
Je nach Größe und Struktur der Mission kann die \ac{OPL} identisch sein mit
\begin{itemize}
	\item der Sektionsführung, falls die Truppstruktur der Mission nur aus einer Sektion besteht
	\item der Zugführung, falls die Truppstruktur der Mission nur aus einem Zug besteht (egal ob Infanteriezug, Panzerzug oder mechanisierte Infanterie). Dies ist die einzige Ausnahme, in der die \ac{OPL} per Short-Range statt Long-Range mit ihren untergeordneten Einheiten kommuniziert.
\end{itemize}
Die OPL hält typischerweise einen sehr großen Abstand zu ihren Truppen - oft bleibt sie auch durchgehend in der Basis.

\section{Sektionsführung / Platoon Lead (PLT)}
\begin{wrapfigure}{r}{0.4\textwidth}
	\vspace{-25pt}
	\centering 
	\includegraphics[width=0.3\textwidth]{./img/truppenordnung/sektionsfuehrung/sektionsfuehrung}
	%\caption{Beispiel einer Sektionsführung}
	\vspace{-90pt}
\end{wrapfigure}	
Die Sektionsführung ist ein Bindeglied zwischen OPL und Zugführung, um die OPL zu entlasten. Einer Sektionsführung sind mindestens zwei, maximal vier Züge unterstellt. Ab drei Zügen in einer Mission ist die Sektionsführung zwingend erforderlich, darunter optional.\\

Die Sektionsführung ist folgendermaßen aufgebaut:
\begin{itemize}
	\item Sektionsführer\,/\,Platoon Lead (PLT): Er leitet die ihm untergeordneten Züge. Kommuniziert wird über Long-Range - entweder über die individuellen Frequenzen der einzelnen Züge oder über die Task-Force-Frequenz (siehe nächstes Kapitel).
	\item Funker / Radio Operator (RO): Übernimmt die Kommunikation zur OPL und anderen Einheiten.
\end{itemize}
Ergänzt werden kann die Sektionsführung bei Bedarf durch:
\begin{itemize}
	\setlength\itemsep{0em}
	\item einen Gefechtssanitäter\,/\,Combat Medic (CM) zur Versorgung im Feld.
	\item einen Fahrzeugführer\,/\,Nahsicherer zur selbstständigen Verlegung
\end{itemize} 
Die Sektionsführung befindet sich typischerweise etwas weiter entfernt hinter den ihr unterstellten Zügen.

\input{./tex/truppenordnung/zugfuehrung/zugfuehrung}
\section{Infanterietrupp / Fireteam (FT)}
\begin{wrapfigure}{R}{0.35\textwidth}
	\vspace{-50pt}
	\centering 
	\includegraphics[width=0.2\textwidth]{../img/truppenordnung/infanterie/infanterie}
	%\caption{Beispiel eines Infantrietrupps}
	\vspace{-90pt}
\end{wrapfigure}
Die Infanterie bildet den Kernbestandteil vieler Missionen. Ein Infanterietrupp ist immer Teil eines Zuges und besteht aus 4 oder 6 Mann. Mögliche Positionen innerhalb eines Infanterietrupps sind:
\vspace{3.5cm}
\begin{longtable}{@{}P{0.4\textwidth}P{0.4\textwidth}@{}}
	\toprule
	Deutsche Bezeichnung & Englische Bezeichnung\\
	\midrule
	Truppführer (TF) & Fireteam Leader (FTL)\\
	Grenadier (GRE) & \\
	Leichter MG-Schütze (LMG) & Automatic Rifleman (AR)\\
	Mittlerer MG-Schütze & Medium Machine Gunner (MMG) \\
	MG-Assistent & Assistant Machine Gunner (AMG)\footnote{notwendig für MMG}\\ 
	Leichter Panzerabwehrschütze & Light Anti Tank (LAT)\\
	Schwerer Panzerabwehrschütze & Heavy Anti Tank (HAT)\\
	Panzerabwehr-Assistent & Assistant Anti Tank (AAT)\footnote{notwendig für HAT}\\ 
	Luftabwehrschütze & Anti-Air (AA)\\
	Pionier & Pioneer (PIO)\\
	Gefechtssanitäter & Combat Medic (CM)\\
	Schütze & Rifleman (RI)\\			
	\bottomrule					
\end{longtable}


Auf eine sinnvolle Einteilung in Buddy"=Teams (z.\,B. bei Positionen, die einen Assistenten erfordern), ist hierbei zu achten.\\
Die Kommunikation erfolgt ausschließlich über Short-Range, der Truppführer schaltet sich über seine Additional-Short-Range auf den Zugkanal auf, um sich mit der Zugführung und den anderen Truppführern im Zug abzusprechen. Die Nummer 2 im Trupp kann sich ebenfalls auf den Zugfunk aufschalten, jedoch nur mithören und nicht funken -- es sei denn, der Truppführer fällt aus und die Nummer 2 übernimmt.

\section{Spezialtrupp}
\includegraphics[width=20mm]{./img/truppenordnung/spezialeinheiten/sf1}\quad
\includegraphics[width=20mm]{./img/truppenordnung/spezialeinheiten/sf2}\\
Spezialtruppen sind infanteristische Einheiten bestehend aus zwei bis sechs Mann mit einem klaren Aufgabenschwerpunkt -- dies kann vom klassischen Zwei"=Mann"=Scharfschützenteam bis zum Sechs-Mann-Kampftauchertrupp gehen. Sie sind die flexibelsten Einheiten innerhalb des TTTs und können entweder autark arbeiten oder im Verbund mit einem anderen Trupp oder Zug. Pro Mission existieren maximal zwei autark operierende Spezialtruppen, im Verbund mit einem Zug maximal einer.\\
Kommunikation erfolgt über Long"=Range, beim Arbeiten im Verbund zusätzlich über die Additional"=Short"=Range (Zugfunk). Kämpfende Einheiten wie z.B. Kommandotrupps oder Kampftaucher, in denen der Truppführer viel Mikromanagement leisten und der Trupp in direkte Feuergefechte verwickelt wird, benötigen zwingend einen separaten Funker. In unterstützenden Einheiten wie z.B. Aufklärungsteams oder Mörserteams, die voraussichtlich nicht in direkte Feuergefechte verwickelt werden, kann (muss aber nicht) der Truppführer die Long"=Range"=Kommunikation mit übernehmen.\\
Mögliche Aufgabenschwerpunkte eines Spezialtrupps sind z.B.:

\begin{itemize}
	\setlength\itemsep{0em}
	\item JTAC-Team
	\item Aufklärungsteam (UAV)
	\item Autonome Kampfeinheit (UGV)
	\item Mörserteam
	\item schwere Feuerunterstützung (Schweres Maschinengewehr (HMG) / Granatmaschinengewehr (GMG))
	\item (schwere) Panzerabwehr / Flugabwehr (falls nicht bereits im Zug vorhanden)
	\item Pionier-Team (falls nicht bereits im Zug vorhanden)
	\item medizinische Versorgung/Unterstützung (falls kein MedEvac in der Mission vorhanden)
	\item Kommandokräfte (Infiltration)
	\item Scharfschützenteam
	\item Kampftaucher
\end{itemize}
Bei entsprechenden Rollen (schwere Waffen, Mörser, etc.) ist auf das Vorhandensein eines entsprechenden Assistenten zu achten.
\section{Logistik}
\includegraphics[width=20mm]{../img/truppenordnung/logistikMedevac/silber}\linebreak
Silber -- Bussard -- Stellt Fahrzeuge und Personal bereit, mit denen Transport und Logistik durchgeführt werden.
\section{MedEvac}
\includegraphics[width=20mm]{../img/truppenordnung/logistikMedevac/weiss}\\
Weiß -- \acf{MedEvac} -- Unterstützt Operationen mit Versorgungs- und Transportkapazität für Verwundete.
\section{Close Air Support}
\includegraphics[width=20mm]{../img/truppenordnung/logistikMedevac/silber}\\
Silber -- Adler -- Stellt Fahrzeuge und Personal bereit,  Gefechtsunterstützung (\ac{CAS}) und Geleitschutz durchgeführt werden.
\input{./tex/truppenordnung/mechanisierteInfanterie/mechanisierte_infanterie}
\input{./tex/truppenordnung/kampfpanzer/kampfpanzer}
\input{./tex/truppenordnung/artillerie/artillerie}