% Dateiformat 
\usepackage[utf8]{inputenc}

% Sprache
\usepackage[ngerman]{babel}

% Silbentrennung
\usepackage[T1]{fontenc}  

% Serifenlose Schrift wie Arial
\usepackage[scaled]{uarial}
\renewcommand*{\familydefault}{\sfdefault}

% Zeilenabstand
\linespread{1.15}
\setlength{\parindent}{0pt} 

% Seitenränder etc
\usepackage[left=3cm,right=3cm,top=2cm,bottom=2cm,bindingoffset=0mm]{geometry}

% Bilder und Grafiken
\usepackage{graphicx}

% Kopf- und Fußzeilen
\usepackage{footmisc}
\usepackage[automark]{scrpage2} 

% Aufzählungen, Tabellen, ...
\usepackage{enumitem}
\usepackage{longtable}
\usepackage{supertabular}
\usepackage{tabularx}
\usepackage{longtable}
\usepackage{colortbl}
\usepackage{multirow}

% Nummerierung von Abbildungen
\usepackage{chngcntr}
\counterwithin{figure}{section}
\counterwithin{table}{section} 

% Schriftsatz
\usepackage{textcomp}
\usepackage{float}

%Abkürzungsverzeichnis
%Optionen:
%PrintOnlyUsed - nur genutzte Abkürzungen werden verwendet
%Withpage - im Abkürzungsverzeichnis wird die Seitenzahl eingefügt (erstes Auftreten der Abkürzung)
%\usepackage[printonlyused,withpage]{acronym}
\usepackage{acronym}

% Farben
\usepackage{xcolor}
\definecolor{blue}{RGB}{0,0,120}
\definecolor{green}{RGB}{0,153,0}
\definecolor{backcolor}{rgb}{0.9,0.9,0.9}
\definecolor{yellow}{RGB}{255,210,0}
\definecolor{gold}{RGB}{218,165,32}
\definecolor{DarkOrchid}{RGB}{92,30,123}

% Hyperlinks in der PDF
\usepackage[colorlinks, linkcolor=black]{hyperref}
\usepackage{url}

\usepackage{pdflscape}
\usepackage{pdfpages}

% Einstellungen Kopf- und Fußzeile
\pagestyle{scrheadings}
\clearscrheadfoot{}
\ofoot[]{\pagemark}
\ifoot[]{TTT}
\cfoot[]{Version 0.8, \today}
\ohead[]{\headmark}
\setheadsepline[\textwidth]{1pt}

% Ebenen Nummerierungen
\setcounter{secnumdepth}{4} %Nummerierungsebenen
\setcounter{tocdepth}{4}   %Ebenen Inhaltsverzeichnis

%Neuer Splatentyp für Tabellen (dient dazu, dass die erste Splate zentiert werden kann)
\newcolumntype{C}[1]{>{\centering\arraybackslash}m{#1}}

%Eigenen Befehl für Newline definieren
\newcommand{\nl}{\newline}

%Inhaltsverzeichnistiefe bestimmen
\setcounter{tocdepth}{1}