\newpage
\section{Funken}
\label{Funken}
\subsection{Anforderungen an einen Funker}
Der Funker stellt im Gefecht die rechte Hand des Truppführers da. Sie sind dazu da,
Informationen zwischen ihrem und anderen Trupps zu übermitteln und
entgegenzunehmen. \\
Dazu müssen sie die Anweisungen des Truppführers verkürzt
übermitteln und Informationen von anderen Trupps an den Truppführer weitergeben. \\
In vielen Trupps besitzt der Funker weiterhin die Aufgaben wie jeder Soldat auch, sei es
die Beobachtung und Sicherung eines Gebietes oder die Aufnahme eines Feuerkampfs.
Dies muss zusätzlich zur Herstellung einer Informationsschnittstelle geleistet werden. \\
Zusammengefasst ist der Funker DAS Bindeglied zwischen den einzelnen Teams und
zuständig für jegliche Kommunikation zwischen diesen. Je präziser die übermittelte
Information, desto besser kann der Empfänger reagieren. Ein guter Funkspruch kann
(virtuelle) Leben retten. \\

\subsection{Funkcheck}
Zu Beginn jeder Mission wird vom OPL – Funker ein kurzer Funkcheck durchgeführt.
Dabei findet sich der Funker auf den verschiedenen Frequenzen ein und führt einen
Funkspruch durch, um zu prüfen, ob sein Gesprächspartner den Funk hören kann. Dies
vermeidet Komplikationen, falls im Gefecht der Funkpartner am Funkverkehr nicht
teilnehmen kann. \\
Nachfolgend findet ein Funkcheck zwischen der OPL und Team \textcolor{gray}{Grau} (Frequenz 34) statt: \\
\textcolor{yellow}{Gelb}: <<\textcolor{gray}{Grau}, hier \textcolor{yellow}{Gelb}, Funkcheck Frequenz 34, kommen!>> \\
\textcolor{gray}{Grau}: Hier \textcolor{gray}{Grau}, bestätige Funkcheck auf Frequenz 34, kommen!>> \\
\textcolor{yellow}{Gelb}: <<Hier \textcolor{yellow}{Gelb}, Funkcheck bestätigt, Ende! \\
\subsection{Anrufarten} 
Das Anrufen stellt im Funkverkehr der Beginn eines jeden Gesprächs dar, hier legt der
Anrufer fest, mit wem er ein Funkgespräch aufbauen möchte. \\
Wir unterscheiden 3 verschiedenen <<Anrufarten>>, den Einzelanruf, den Reihenanruf und
den Sammelanruf. Ein Anruf baut sich, je nach <<Anrufart>> anders auf. \\
\subsection{ Anrufarten}
Das Anrufen stellt im Funkverkehr der Beginn eines jeden Gesprächs dar, hier legt der Anrufer fest, mit wem er ein Funkgespräch aufbauen möchte. Wir unterscheiden 3 verschiedenen <<Anrufarten>>, den Einzelanruf, den Reihenanruf und den Sammelanruf. Ein Anruf baut sich, je nach <<Anrufart>> anders auf. \\

\subsubsection{Einzelanruf}
Beim Einzelanruf, gibt es einen Sender und genau einen Empfänger, \\
Gezeigt werden Beispiele mit Team \textcolor{blue}{Blau} als Sender und Team \textcolor{green}{Grün} als Empfänger: \\
Team \textcolor{blue}{Blau}: <<\textcolor{green}{Grün}, hier \textcolor{blue}{Blau} kommen!>> \\
Anschließend muss das Team, das angerufen wurde, auch antworten, um zu signalisieren, dass es bereit für einen Funkspruch ist: \\
Team \textcolor{green}{Grün}: <<Hier \textcolor{green}{Grün}, kommen!>> \\
Das Schlüsselwort <<kommen>> signalisiert hierbei, dass der Gesprächspartner eine Antwort erwartet und das der Funkspruch zu Ende ist, der Empfänger also antworten kann. \\

\subsubsection{Reihenanruf}
Beim Reihenanruf gibt es einen Sender und mehrere Empfänger, die Empfänger antworten dabei in der Reihenfolge, in der sie angerufen wurden. \\
Sollte ein Team nicht antworten, wird das Team nach 5 Sekunden übersprungen und das nächste Team in der Reihe reagiert auf den Anruf. \\
Beispiel mit Team \textcolor{green}{Grün} als Sender und Team \textcolor{blue}{Blau} und \textcolor{gold}{Gold} als Empfänger. \\
Team \textcolor{green}{Grün}: <<\textcolor{blue}{Blau}, \textcolor{gold}{Gold}, hier \textcolor{green}{Grün}, kommen!>> \\
Team \textcolor{blue}{Blau} wurde als erstes angerufen, antwortet also auch zuerst, danach kommt Team \textcolor{gold}{Gold}: \\
Team \textcolor{blue}{Blau}: <<Hier \textcolor{blue}{Blau}, kommen!>> \\
Team \textcolor{gold}{Gold}: <<Hier \textcolor{gold}{Gold}, kommen!>> \\

\subsubsection{Sammelanruf}
Beim Sammelanruf werden alle teilnehmenden Teams von einem Sender angerufen. Die Empfänger antworten dazu wie beim Reihenanruf, allerdings in einer vorher festgelegten Reihenfolge. Im Falle des TTT wird die Frequenz, auf dem die Teams sich befinden, auch als Reihenfolge verwendet, das Team auf der niedrigsten Frequenz beginnt mit der Antwort bis zum Team mit der höchsten Frequenz. \\
Beispiel Team \textcolor{yellow}{Gelb} ruft alle Teams an: \\
\textcolor{yellow}{Gelb}: <<\textcolor{yellow}{Gelb} an alle, kommen!>>. \\
\begin{center}
– Antworten – \\
\end{center}
\textcolor{green}{Grün}: <<Hier \textcolor{green}{Grün}, kommen!>>. \\
\textcolor{gold}{Gold}: <<Hier \textcolor{gold}{Gold}, kommen!>>. \\
\textcolor{gray}{Grau}: <<Hier \textcolor{gray}{Grau}, kommen!>>. \\
Weiß: Hier Weiß, kommen!>>. \\
\textcolor{DarkOrchid}{Adler}: Hier \textcolor{DarkOrchid}{Adler}, kommen!>>. \\
Bussard: <<Hier Bussard, kommen!>>. \\
Panther: <<Hier Panther, kommen!>>. \\

\subsection{Gesprächsverlauf}
Während des Gesprächs sollte darauf geachtet werden, sich auf das absolut notwendigste zur Informationsübermittlung zu beschränken. Pausen oder überflüssige Ausschweifungen während des Gesprächs behindern nur die Kommunikation und kosten wertvolle Zeit. Während der Mission sollten regelmäßig Statusberichte an die Operationsleitung übermittelt werden, zum Beispiel nach einem Feuergefecht oder wenn der Auftrag erledigt wurde. Hier sollte angegeben werden, wie der Status der aktuellen Aufgabe ist, ob es Verwundete im Team gibt und wie der Munitionsstand ist. Bei der Übermittlung von Informationen ist es hilfreich, wenn der Funkpartner die Informationen noch einmal wiederholt, damit wird sichergestellt, dass die Informationen
verstanden wurden. \\
Beispiel (Anruf erfolgte bereits) \textcolor{gold}{Gold}: \\
\textcolor{gold}{Gold}: <<Hier \textcolor{gold}{Gold}, haben Feindpatrouille gesichtet, 4 Mann bei Grid 054 –Strich - 168, kommen!>> \\
\textcolor{yellow}{Gelb}: <<Hier \textcolor{yellow}{Gelb}, ich wiederhole: Feindpatrouille mit 4 Mann bei Grid 054 – Strich - 168, kommen!>> \\
\begin{center}
– Gespräch wird beendet –
\end{center}
Des weiteren ist es wichtig, bei jedem Funkspruch seinen Teamnamen zu nennen, um eindeutig identifizierbar zu bleiben. Dies geschieht durch <<Hier <Teamname>, ….>> zu Beginn jedes Funkspruchs. \\
Sollte mehrere Informationen während eines Funkspruchs durchgegeben werden, die inhaltlich voneinander getrennt sind, kann das Schlüsselwort <<Trennung>> verwendet werden. \\
Beispiel: \\
\begin{center}
- Anruf erfolgte bereits -
\end{center}
\textcolor{gray}{Grau}: <<Hier \textcolor{gray}{Grau}, haben Feindkontakt ausgeschaltet, keine Verwundeten, Auftrag ausgeführt, Trennung - erwarten weitere Befehle, kommen!>> \\
Weiterhin kann <<Trennung>> verwendet werden, um mehrere Befehle an mehrere Teams in einen Funkspruch zu erteilen, z.B. bei einem Reihen – oder Sammelanruf. \\
Beispiel: \\
\textcolor{green}{Grün}: <<\textcolor{blue}{Blau}, \textcolor{red}{Rot}, hier \textcolor{green}{Grün}, kommen!>> \\
\textcolor{blue}{Blau}: <<Hier \textcolor{blue}{Blau}, kommen!>> \\
\textcolor{red}{Rot}: <<Hier \textcolor{red}{Rot}, kommen!>> \\
\textcolor{green}{Grün}: <<Hier \textcolor{green}{Grün}, \textcolor{blue}{Blau} rückt weiter die Straße vor und sichert die Häuser, Trennung, \textcolor{red}{Rot} sichert die rechte Flanke von \textcolor{blue}{Blau}, kommen!>> \\

\subsection{Gespräch beenden}
Für das Beenden eines Funkgesprächs gibt es eine große Regel: Derjenige, der ein Gespräch eröffnet, muss es auch beenden. Damit wird vermieden, dass der Gesprächspartner abgewürgt wird, obwohl dieser noch Fragen oder Befehle an seinen Funkpartner abgeben will. \\
Falsches Beispiel: \\
\textcolor{yellow}{Gelb} an \textcolor{green}{Grün}(Anruf erfolgte bereits): <<Hier \textcolor{yellow}{Gelb}, rücken Sie weiter die Straße vor, bis sie die Kreuzung erreichen, kommen!>>  \\
\textcolor{green}{Grün} an \textcolor{yellow}{Gelb}: <<Hier \textcolor{green}{Grün}, ich wiederhole: rücken weiter die Straße vor, bis die Kreuzung erreicht wurde, Ende!>> \\
Hier hat \textcolor{green}{Grün} einfach das Gespräch beendet, ohne \textcolor{yellow}{Gelb} die Gelegenheit zu geben, weitere Informationen zu übermitteln, \textcolor{yellow}{Gelb} müsste nun erneut anrufen, was unnötig Zeit kostet. \\
Richtiges Beispiel: \\
\textcolor{yellow}{Gelb} an \textcolor{green}{Grün}(Anruf erfolgte bereits): <<Hier \textcolor{yellow}{Gelb}, rücken Sie weiter die Straße vor, bis sie die Kreuzung erreichen, kommen!>> \\
\textcolor{green}{Grün} an \textcolor{yellow}{Gelb}: <<, Hier \textcolor{green}{Grün}, ich wiederhole: rücken weiter die Straße vor, bis die Kreuzung erreicht wurde, kommen!>> \\
\textcolor{yellow}{Gelb} an \textcolor{green}{Grün}: <<Hier \textcolor{yellow}{Gelb},verstanden, Ende!>> \\

\subsection{An – und Abmelden im Funkkreis} \label{AnAbmeldenImFunk}
Wenn ein Funker den Funkkreis wechselt, um beispielsweise Informationen an die OPL zu übergeben, dann muss dieser sich vorher anmelden. Sobald die Informationen übermittelt wurden, meldet sich der Funker vom Funkkreis ab und wechselt auf seinen Funkkreis. \\

\subsubsection{Anmelden im Funkkreis}
Das Anmelden im Funkkreis wird vor dem eigentlichen Anruf gesetzt. Mit der Anmeldung wird bekanntgegeben, dass sich nun ein weiterer Funkpartner im Funkkreis befindet.
Im folgenden Beispiel meldet sich Team \textcolor{gray}{Grau} auf den Kanal von Team \textcolor{yellow}{Gelb} (OPL) an: \\
\textcolor{gray}{Grau}: <<\textcolor{yellow}{Gelb}, hier \textcolor{gray}{Grau}, \textcolor{gray}{Grau} meldet sich auf dem Funkkreis an, kommen!>> \\
\textcolor{yellow}{Gelb}: <<Hier \textcolor{yellow}{Gelb}, kommen!>> \\ 
\begin{center}
– weiterer Gesprächsverlauf –
\end{center}
\subsubsection{Abmelden im Funkkreis}
Wenn das Gespräch beendet ist, kehrt der Funkpartner, der sich auf dem Funkkanal angemeldet hat, wieder zurück in seinen eigenen Funkkreis, wenn nicht anders befohlen. Dabei gibt dieser die Information, dass er sich abmeldet. \\
Beispiel: \\
\textcolor{yellow}{Gelb}: <<Hier \textcolor{yellow}{Gelb}, rücken Sie weiter vor bis Kreuzung, kommen!>> \\
\textcolor{gray}{Grau}: <<Hier \textcolor{gray}{Grau}, ich wiederhole, rücken vor bis Kreuzung, - Trennung – Melde mich vom Funkkreis ab/ ich melde mich ab, Ende!>> \\

\subsection{Bildung eines Funkkreises}
Wenn zwei oder mehrere Trupps über einen längeren Zeitraum zusammen agieren, ist es sinnvoll einen gemeinsamen Funkkreis zu bilden, dies geschieht meist auf Befehl der Operationsleitung. Die Trupps erhalten dabei eine neue Frequenz, behalten aber ihren Rufnamen. Sobald die Aufgabe, für dessen Zweck der Funkkreis gebildet wurde, erfüllt wurde, wird der Funkkreis eigenständig aufgelöst. Die Operationsleitung muss dabei in Kenntnis gesetzt werden, dass die beteiligten Trupps wieder auf ihren ursprünglichen Frequenzen zu finden sind, siehe dazu \autoref{AnAbmeldenImFunk}. \\
\textcolor{yellow}{Gelb} funkt \textcolor{gray}{Grau} und \textcolor{DarkOrchid}{Adler} an und befiehlt Bildung eines Funkkreises \\
\begin{center}
- Anruf erfolgte bereits -
\end{center}
\textcolor{yellow}{Gelb}: <<Hier \textcolor{yellow}{Gelb}, Team \textcolor{gray}{Grau} bildet zusammen mit Team \textcolor{DarkOrchid}{Adler} einen Funkkreis, um Stadt von Feindkräften zu säubern, Frequenz ist die fünf – zwo, wiederhole fünf - zwo kommen.>> \\
– gleiche Information an Team \textcolor{DarkOrchid}{Adler} - \\
– Nachdem Auftrag ausgeführt wurde - \\
\textcolor{gray}{Grau}: <<Hier \textcolor{gray}{Grau}, Stadt wurde von Feindkräften gesäubert, Funkkreis wurde aufgelöst,\textcolor{gray}{Grau} befindet sich nun wieder auf Frequenz drei – drei, kommen.>> \\

\subsection{Funken im Gefecht}
Da längere Funkgespräche während eines Gefechts eine Behinderung für den Trupp darstellen, wird das Protokoll während eines Gefechts gekürzt. Das bedeutet, dass der Anruf und die Übermittlung der Informationen zusammen in einem Funkspruch durchgeführt werden. Im Beispiel befindet sich \textcolor{gold}{Gold} im Gefecht und ruft die OPL an: \\
\textcolor{gold}{Gold}: <<\textcolor{yellow}{Gelb}, hier \textcolor{gold}{Gold}, melde mich im Funkkreis an – Trennung - stehen unter schwerem \\
Beschuss, benötigen Unterstützung durch CAS, kommen!>> \\

\subsection{Befohlener Frequenzwechsel}
Bei diversen Situationen wird von der Operationsleitung ein Frequenzwechsel auf eine vorher festgelegte Frequenz befohlen. Dabei wechseln sämtliche angesprochene Teams ohne Bestätigung auf diese Frequenz. Beim TTT wird diese Praxis hauptsächlich bei Konvoibildung eingesetzt. Dabei weist die Operationsleitung einen Frequenzwechsel auf die Konvoifrequenz an. \\
Beispiel: \\
\textcolor{yellow}{Gelb} (OPL): <<Frequenzwechsel auf Konvoifrequenz, Hier \textcolor{yellow}{Gelb}, Ende!>> \\

\subsection{Korrigieren und Annullierung eines Funkspruchs}
Natürlich kann es vorkommen, dass während eines Funkspruchs Fehler bei der Übermittlung der Informationen auftreten oder sich der Grund des Anrufs erledigt hat. Dazu kann man einen Funkspruch korrigieren oder annullieren, damit der Funkpartner über die Änderungen in Kenntnis gesetzt werden kann. \\

\subsubsection{Korrektur eines Funkspruchs}
Korrektur eines Funkspruchs, mitten im Funkspruch: \\
\begin{center}
– Anruf erfolgte bereits –
\end{center}
\textcolor{gray}{Grau}: <<Hier \textcolor{gray}{Grau}, melden T – 90 auf Grid null zwo sechs – ich korrigiere/berichtige: null drei drei – Strich - fünf zwo sieben, kommen!>> \\
Korrektur eines Funkspruchs, nach dem Funkspruch: \\
\begin{center}
– Anruf erfolgte bereits –
\end{center}
\textcolor{gray}{Grau}: <<Hier \textcolor{gray}{Grau}, melden T – 90 auf Grid null zwo drei – Strich – fünf zwo sieben, kommen!>> \\
\begin{center}
– Fehler wird bemerkt – \\
\end{center}
Hier muss angegeben werden, welche Information genau korrigiert wird(Feindstärke, Grid, Verwundetenzahl usw.). \\
\textcolor{gray}{Grau}: <<Ich korrigiere/berichtige Grid: null drei drei – Strich - fünf zwo sieben, kommen!>> \\

\subsubsection{Annullierung eines Funkspruchs}
Annullierung eines Funkspruchs, mitten im Funkspruch: \\
\begin{center}
– Anruf erfolgte bereits –
\end{center}
\textcolor{gold}{Gold}: <<Hier \textcolor{gold}{Gold}, melden Feindpatrouille, 6 Mann auf Grid... - Trennung – ich annulliere diesen Funkspruch, kommen!>> \\
Annullierung eines Funkspruchs, nach dem Funkspruch: \\
\begin{center}
– Anruf erfolgte bereits – \\
\end{center}
\textcolor{gold}{Gold}: <<Hier \textcolor{gold}{Gold}, melden Feindpatrouille, 6 Mann auf Grid null drei drei – Strich – fünf zwo sieben, kommen!>> \\
\textcolor{gold}{Gold}: <<\textcolor{yellow}{Gelb}, hier \textcolor{gold}{Gold}, annullieren vorherigen Funkspruch, kommen!>> \\

\subsection{Unterbrechen eines Funkgesprächs}
Es kann vorkommen, dass man sich in einem anderen Funkkreis anmelden möchte, um eine wichtige Mitteilung zu übergeben, bemerkt aber dabei, dass bereits ein Funkgespräch auf dieser Frequenz stattfindet. Hierbei sollte der Funker darauf achten, ob die Wichtigkeit seiner Mitteilung die Wichtigkeit des Gesprächs überwiegt. Bei absoluten Notfällen ist es also erlaubt, das bereits stattfindende Gespräch zu unterbrechen und ein eigenes Funkgespräch zu eröffnen. Dies wird mit den Schlüsselwörtern <<Break, Break>> eingeleitet. \\
Die beiden Gesprächspartner, die dadurch unterbrochen werden, sollten daraufhin ihr Gespräch unmittelbar ohne jedwede Bestätigung einstellen und auf die Nachricht des <<Unterbrechers>> warten. \\
Beispiel(\textcolor{gray}{Grau} wechselt auf die Frequenz von \textcolor{DarkOrchid}{Adler}, um CAS anzufordern, unterbricht dabei Konversation zwischen \textcolor{gold}{Gold} und \textcolor{DarkOrchid}{Adler}):\\
\textcolor{gold}{Gold}: <<Hier \textcolor{gold}{Gold}, [Inhalt der Nachricht], kommen. \\
\textcolor{gray}{Grau}(unterbricht Gespräch): <<Break, Break, \textcolor{DarkOrchid}{Adler}, hier \textcolor{gray}{Grau}, benötigen dringend Luftunterstützung, stehen unter schwerem Beschuss und haben Verluste, kommen.>>  \\

\subsection{Halten eines Funkgesprächs}
Sollte es während eines Funkgesprächs vorkommen, dass eine Weiterführung für kurze Zeit nicht möglich ist und einer der Gesprächspartner eine kurze Bedenkzeit benötigt, um das Gespräch fortzuführen, ist es möglich, das laufende Gespräch zu halten. Dazu bleiben beide Gesprächspartner weiterhin im gleichen Funkkreis, da das Gespräch noch nicht beendet wurde. \\
Beispiel(\textcolor{gold}{Gold} übermittelt Informationen an \textcolor{gray}{Grau}, \textcolor{gray}{Grau} benötigt Bedenkzeit, um zu antworten): \\
\textcolor{gold}{Gold}: <<Hier \textcolor{gold}{Gold}, [Inhalt der Nachricht], kommen.>> \\
\textcolor{gray}{Grau}: <<Hier \textcolor{gray}{Grau}, Standby/ Warten Sie. \\
\begin{center}
– Nach kurzer Wartezeit - \\
\end{center}
\textcolor{gray}{Grau}: <<Hier \textcolor{gray}{Grau}, [Inhalt der Nachricht], kommen.>> \\

\subsection{Anfrage für Freigabe von Sondereinsätzen}
Während einer Mission können die Teams mit entsprechender Freigabe selbstständig Unterstützung durch andere Teams anfordern. Diese Freigabe können sie sich unter gegebenen Umständen bei der OPL einholen. Die OPL kann die Freigabe dann entweder erteilen oder auch verweigern, wenn z.B. bei einer CAS – Anfrage unklar ist, ob es Luftabwehrstellungen gibt. Die Freigabe kann missionsgebunden sein, d.h. die Freigabe gilt für die gesamte Mission, oder die Freigabe ist auftragsgebunden, d.h. die Freigabe wird nur für einen speziellen Auftrag erteilt, z.B. für die Bekämpfung eines Panzers mit CAS. \\
Beispiel(Trupp \textcolor{gray}{Grau} erbittet Freigabe für CAS bei der OPL): \\
\begin{center}
– Anruf erfolgte bereits -
\end{center}
\textcolor{gray}{Grau}: <<Hier \textcolor{gray}{Grau}, erbitten Freigabe für CAS zur Säuberung der Stadt, Gebiet ist aufgeklärt, keine Luftabwehrstellungen gesichtet, kommen.>> \\
\textcolor{yellow}{Gelb}: <<Hier \textcolor{yellow}{Gelb}, so verstanden, Freigabe für CAS zur Säuberung der Stadt ist erteilt, kommen.>>\\

\subsection{Funkverkehr bei schlechtem Verständnis}
Sollte es vorkommen, dass die Funkteilnehmer zu weit voneinander entfernt sind und kein störungsfreies Funkgespräch zustande kommt, kann ein Funkteilnehmer eine Aufforderung zum zweimaligen Sprechen ausgeben. Dabei wird jeder wichtige Teil eines Satzes im Funkgespräch wiederholt, dies gilt auch für den Anruf und die Beendigung des Funkverkehrs. \\
Beispiel(Funkverkehr zwischen \textcolor{green}{Grün} und \textcolor{DarkOrchid}{Adler}): \\
\textcolor{green}{Grün}: <<\textcolor{DarkOrchid}{Adler}, hier \textcolor{green}{Grün}, \textcolor{green}{Grün} meldet sich auf dem Funkkreis an, kommen.>> \\
\textcolor{DarkOrchid}{Adler}: <<\textcolor{green}{Grün}, hier \textcolor{DarkOrchid}{Adler}, verstehen sie schlecht, sprechen sie zwomal, ich wiederhole, sprechen sie zwomal, kommen>> \\
\textcolor{green}{Grün}: <<Hier \textcolor{green}{Grün}, hier \textcolor{green}{Grün}, so verstanden, so verstanden, Trennung, Trennung, benötigen Aufklärung, benötigen Aufklärung, südlich unserer Position, südlich unserer Position, kommen, kommen.>> \\



