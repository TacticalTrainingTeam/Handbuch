\section{Vorwort}
\subsection{Über dieses Dokument}
	Entstanden aus dem ursprünglichen, kompakten Leitfaden zur Allgemeinen Grundausbildung im \ac{TTT} soll dieses Dokument sowohl Grünschnäbeln als auch Alten Hasen zum Erlernen und Festigen aller relevanten theoretischen Grundlagen des taktischen Spiels im \ac{TTT} dienen. Es ist Leitfaden, Referenzwerk und 		Messlatte in einem Dokument vereint. Wie jedes Schriftstück zur Ausbildung sollte es \textit{aktiv} gelesen werden: der Leserin oder der Leser sollte über die vermittelten Inhalte nachdenken, sie hinterfragen, diskutieren wo nötig und gewissenhaft einprägen.\\
	Dieses Dokument lebt, es wird weiterentwickelt, überarbeitet, erweitert, gekürzt -- Anmerkungen, Vorschläge und Rückmeldungen konstruktiver Art werden nicht nur gewünscht, sondern gefordert. Ein hoher Standard im Spiel beginnt mit einem hohen Standard und Anspruch an alle Unterlagen, und jeder Spieler im \ac{TTT} ist nicht nur ständigen Verbesserung und Erweiterung seiner eigenen Fähigkeiten verpflichtet, sondern auch zur Verbesserung und Erweiterung des vermittelten Wissens. Ein aufmerksamer, geübter Schütze mag den Ausgang eines Feuerkampfs für sich entscheiden - der virtuelle Krieg wird jedoch durch Strategie, Taktik und das Wissen um das \textit{WIE} entschieden, das aus den Erfahrungen von Fehlschlägen und Triumphen herrührt. Der erste Schritt den du, lieber Leser, auf dem Weg zu einem erfolgreichen Spieler im \ac{TTT} leisten musst, ist über diese Aussage zu meditieren und sie zu verstehen.\\

\textit{Train hard, play smart!}

\newpage
\subsection{Die Grundsätze des TTT}
%TODO: Schreiben