%TODO: Links, Erklärungen, Einleitung

\section{Das AGA-Handbuch}

\subsection{Über die Grundausbildung}

\subsection{Organisation der Trupps}


\subsubsection{Organisation und Befehlshierarchie}

%TODO: Erklärung warum und wie es funktioniert
	Das Ziel jeder militärischen Organisation ist es, die gesetzten Ziele schnell und effizient zu erreichen: wie kleine, fein abgestimmte Zahnräder sollen Befehlender und Befehlsgeber miteinander interagieren und auf rapide Veränderungen ebenso rapide reagieren. Dazu ist die Einhaltung der Befehlshierarchie, der Funk- und Meldediziplin absolut notwendig.\\
	Im englischsprachigen Raum existiert der Begriff \textit{chain-of-command}, welcher im deutschen Sprachraum äquivalent mit den Begriffen \textit{Dienstweg} und, besonders im militärischen, \textit{Befehlshierarchie} ist. Man versteht darunter, wie und an wen Befehle ausgegeben und angenommen werden. Ganz wichtig ist: Sofern nicht explizit anders geregelt (Grün: "Schwarz meldet Sichtungen direkt an \ac{OPL}") werden alle Befehle nur an direkt Unterstellte ausgegeben und Meldungen nur an direkte Vorgesetzte gemacht. Als Beispiel sei genannt, dass die Zugführung einem Soldaten im Regelfall nicht einen expliziten Befehl geben ("Sicherungsbereich Nord-nord-ost") wird, dafür ist der Truppführer zuständig. Dies geschieht, um eine klare Übersicht und Kontrolle auch in komplizierten und unübersichtlichen Situationen zu haben - und fußt auf Vertrauen. Dem Vertrauen, dass der Befehlende nur solche Befehle ausgibt, die der Befehlsempfänger unter der Berücksichtigung seiner momentanen Lage (Ausrüstung, Position, Zustand, ...) realistischer Weise schaffen kann. Umgekehrt vertraut ein Befehlender darauf, dass ein solcher Befehl ausgeführt und der Ausgang gemeldet wird: Kommt es beispielsweise zu momentan unauflösbaren Schwierigkeiten, werden diese gemeldet. \\
	Jeder sollte dabei im Hinterkopf behalten, dass...
		\begin{itemize}
			\item ... ein schlechter Befehl manchmal besser ist als kein Befehl.
			\item ... mit wenigen Ausnahmen jeder Spieler auch ein Mensch ist. 
			\item ... mit Herz und Verstand befehligt und gehorcht wird. 
			\item ... für jede Diskussion die \ac{AAR} da ist.
			\item ... das Ziel ist, jederzeit immer besser zu werden.
			\item ... eine "Scheißegal"-Attitüde dazu führen muss, dass du den Server verlässt.
		\end{itemize}

\subsubsection{Die Trupps}

%TODO: Ausführliche Beschreibung der Aufgabengebiete

\subsubsection{Uniformen und Zeichen}
	In den meisten Fällen lassen sich Mitglieder des \ac{TTT} an den Zeichen auf ihren Uniformen erkennen und identifizieren:
		\begin{itemize}
			\item Patch auf dem linken Oberarm in Truppfarbe und mit dem entsprechenden Logo
			\item Auf der Rückseite des Helms ist die Nummer des Soldaten im Trupp angebracht
			\item Auf jedem Handschuh befindet sich das Logo des \ac{TTT} in der entsprechenden Truppfarbe
		\end{itemize}

\subsection{Dein Buddy und du}

%TODO: Wichtige Lektionen einfügen
\centerline{\textit{Dies ist mein Buddy. Davon gibt es viele, aber dieser gehört mir.}}
	Die Aussage einer Erzieherin \textit{"So, und jetzt nimmt jeder die Person links von ihm an die Hand"} fasst am besten zusammen, für was ein Buddyteam steht: Man passt gegenseitig aufeinander auf, sodass man nicht verloren geht. Egal ob beim Tauchen, bei einer Ralley, im Flugzeug, beim Pair-Programming oder im Scharfschützenteam - die Stärke der Kleinsten aller Gruppen besteht darin, dass zwei Personen an einem Strang ziehen und sich dabei gegenseitig unterstützen können. \\
	Das Buddyteam im \ac{TTT} ist viele Dinge: die kleinste organisatorische Einheit, das solideste Fundament jedes Trupps, die beste Versicherung eines einzelnen Spielers das Ende der Mission zu sehen. Buddyteams werden üblicherweise so gebildet, dass sie einer speziellen Funktion folgen können: Sei dies als Maschinengewehrcrew, Fahrzeugbekämpfung, Aufklärung, Führung oder jeder anderen denkbaren Aufgabe. Buddyteams haben auch Nachteile. Darunter ist beispielsweise, dass sie üblicherweise zufällig für eine Mission gebildet werden, die beiden Personen sich nicht notwendigerweise verstehen und ähnliches. Regelmäßiges Spielen und Trainingkann diese Probleme eindämmen, jedoch niemals ganz. \\
	Das zentrale Element eines solchen Teams bist immer \textit{DU} -- Dein Streben muss es ständig sein, ein besserer Buddy zu werden. Nach der AAR nicht nur Vorgesetzte und erfahrenere Spieler, sondern auch den für die Mission zugeteilten Buddy zu fragen was man verbessern kann sollte zum guten Ton eines \ac{TTT}-Spielers gehören. Auch während der Mission gibt es ein paar Dinge, über die man sich jederzeit im Klaren sein sollte:
		\begin{itemize}
			\item Position meines Buddys
			\item Zustand meines Buddys
			\item Ausrüstung, momentaner Munitionsstand meines Buddys
			\item Grobe Blickrichtung oder Sicherungsrichtung meines Buddys
			\item ...
		\end{itemize}
	Auch hier hilft nur eins: Erfahren, denken, üben. So hilft es beispielsweise nichts, einfach so loszustürmen um seinen Buddy zu bergen und dabei selbst schwer verwundet zu werden. 

\subsection{Der Trupp}

Fasst man mindestens zwei Buddyteams zusammen, so entsteht ein klassischer Kommando-Trupp wie man ihn beispielsweise für den Häuserkampf oder Spezialeinsätze benötigt. Wie auch beim Buddyteam entsteht dabei die Stärke eines Trupps dadurch, dass sich die einzelnen Teams gegenseitig unterstützen und sichern können. Je besser ein Trupp aufeinander eingespielt ist, umso effizienter wird er. Und gerade bei einem 4-Mann-Trupp ist ein hohes Maß an Übung und Einspielen erforderlich, da bereits der Ausfall eines einzelnen Soldaten den Trupp signifikant an der Erfüllung seines Auftrags hindern kann. \\
	Ein Trupp aus drei Buddyteams ist nicht nur robuster, sondern auch taktisch vielfältiger einsetzbar: so kann beispielsweise ein Buddyteam unter der Sicherung der beiden anderen Teams vorrücken. Er ist der Standardtrupp, der im \ac{TTT} Verwendung findet. Wie in solcher Trupp aussehen kann ist nachfolgend aufgeführt:
		\begin{itemize}
			\item Buddy-Team 1, “Führung” 
			\begin{itemize}	
				\item Nr.1: Truppführer -  Führt den Trupp durch klare, vorbildliche Führung von vorn
				\item Nr.2: Stellv. Truppführer - Unterstützt den Truppführer indem er Funk oder ähnliche Aufgaben übernimmt
			\end{itemize}
		\end{itemize}

		\begin{itemize}
			\item Buddy-Team 2, “Schwere Waffen” 
			\begin{itemize}
				\item Nr.3: MG-Assistent - Trägt zusätzliche Munition für das MG, hilft beim einschießen auf die Ziele, lenkt das MG, trägt Munition, unterstützt mit Nahsicherung
				\item Nr.4: MG-Schütze - Trägt das MG und führt den Feuerkampf
			\end{itemize}
		\end{itemize}

		\begin{itemize}
			\item Buddy-Team 3, “Spezialisten” 
			\begin{itemize} 
				\item Nr.5: Spezialist 1 - Übernimmt zusammen mit seinem Buddy Spezialaufgaben wie beispielsweise Grenadier, \ac{AA}/\ac{AT}, \ac{EOD}, Sanitäter, Designated Marksman
				\item Nr.6: Spezialist 2 - Übernimmt zusammen mit seinem Buddy Spezialaufgaben wie beispielsweise Grenadier, \ac{AA}/\ac{AT}, \ac{EOD}, Sanitäter, Designated Marksman
			\end{itemize}
		\end{itemize}
Es ist nicht ungewöhnlich, dass man über seine Nummer im Trupp oder die Nummer des Buddyteams angesprochen wird. Es ist daher von Vorteil, sich mit Stift und Papier zu merken in welchem Buddyteam man ist und welche Nummer man selbst und die anderen Leute haben.



\newpage

\subsection{Die Befehle – die wichtigsten Kommandos im TTT}

	Die häufigsten Kommandos eines Trupp- oder Zugführers \\

\subsubsection{Kommando lautet <<Deckung!>>}

	Die Deckung ist ein situationsbedingter Aufenthaltsort \\ 

	Wird man vom Feind überrascht, ist der Soldat angewiesen, sofort Deckung im nahen Umfeld (Radius 20 Meter) zu suchen – ohne die Trupp-Struktur aufzulösen. Dabei gilt zu beachten, dass die Buddy-Teams immer im Verbund bleiben. \\

	Als Deckung zählt alles, was als beschusssicher gilt: Häuser, Felsen, Mauern, Senken. Ist keine Deckung in Laufweite, wird sofort Bodenlage eingenommen und auf den Feind ausgerichtet. Bei indirektem Beschuss (Granaten, etc.) wird ein besonders weiter Abstand zwischen den Kameraden gesucht.\\

	Der Befehl vom Truppführer lautet <<Deckung!>> oder als ausführlicheres Beispiel <<Deckung, 6 Uhr, hinter der Mauer!>> \\

\subsubsection{Kommando: <<Bekämpfen …!>>}

	Der Soldat soll einen erkannten Feind bekämpfen. Hierbei ist es wichtig, genau abzuwägen, wie viel Einweisung benötigt wird, damit das Kommando erfolgreich ausgeführt werden kann. Der Soldat meldet bei erfolgreicher Absolvierung des Auftrags <<Gegner, YXZ, bekämpft!>> \\

\subsubsection{Kommando: <<Feuer frei !>>}

	Das ist die direkte Anweisung, JETZT zu schießen. Kann ergänzt werden mit <<Feuern, 3 Salven>> oder sonstigen Einschränkungen. \\

\subsubsection{Kommando: <<Unterdrückungsfeuer!>>}

	Das ist die direkte Anweisung, in die generelle Richtung des Gegners zu schießen, um ihn in die Deckung zu zwingen und ihn an Gegenfeuer oder Manövern zu hindern. \\

\subsubsection{Kommando <<Gezieltes Feuer!>>}

	Das ist die direkte Anweisung, den Gegner mit gezielten Salven oder Einzelschüssen direkt zu treffen – also das Gegenteil vom Unterdrückungsfeuer. In der Regel wird bei der Feuerfreigabe erwartet, dass gezieltes Feuer abgegeben wird. \\

\subsubsection{Kommando <<Feuerfreigabe erteilt>>}
	Grundsätzliche Feuerfreigabe. Das heißt, sobald Gegner in Sicht und der Feuerkampf ist aussichtsreich, darf geschossen werden. Gilt bis zum Widerruf. \\

\subsubsection{Kommando <<Feuer nur auf Freigabe>>}
	Vor dem Schuss muss Feuerfreigabe angefordert werden. \\

\subsubsection{Kommando: <<Stopfen!>>}
	Der Soldat stellt sofort das <<Schießen>> ein. \\

\subsubsection{Kommando lautet <<Tarnen!>>}

	Das bedeutet, dass man sich in den Schutz von Büschen, Bäumen oder Sonstigem begeben soll. Wichtig beim Tarnen ist, dass man vom Gegner nicht erkannt wird. Dabei gilt es die Grundregeln zu beachten (Sky-lineing, etc.). Gras taugt nicht als Tarnung! \\

\subsubsection{Kommando: <<Bezieht Stellung bei …!>>}

	Die Stellung ist ein selbst gewählter Aufenthaltsort. \\

	Die Stellung ist idealerweise ein gewählter und geschützter Aufenthaltsort (Haus, Senke, Felsen, Mauer) und lässt sich leicht verteidigen. Die Soldaten haben dabei maximale Deckung eingenommen und sichern. \\

	Der Truppführer gibt den Befehl aus, um einen Aufenthaltsort festzulegen, wo sich der Trupp aufhalten soll. Daraus folgt automatisch, dass der Trupp den Ort erreicht und vorläufig sichert. Der Sicherungsschwerpunkt liegt automatisch auf vermutete Feindstellungen. Mit dem Zusatz <<gedeckt>> betont er, dass der 		Trupp eine Aufklärung durch den Feind unter allen Umständen vermeiden soll. In der Praxis: Nähern in die gedeckte Stellung unter Sicht- und Deckungsschutz. Dort in die Stellung getarnt beziehen und halten. In der Stellung erfolgt vom Truppführer:
		\begin{itemize}
			\item Kontaktaufnahme mit dem Zugführer 
			\item Geländetaufe 
			\item Verteilung von Feuerbereichen 
			\item Vorausschauende Planung (Wegplanung, Korridore, weitere Stellungen) 
			\item Kommando: <<Bezieht gedeckte Stellung bei …!>> 
		\end{itemize}

\subsubsection{Kommando: <<Wechselt die Stellung!>> (Nein, nicht was ihr denkt …)}

	Der taktische Stellungswechsel ist ein eigenständiger Positionswechsel in eine andere Stellung – beispielsweise, weil man befürchtet, die Stellung ist vom Feind bereits aufgeklärt wurde und damit Feindfeuer erwartet. Das gilt besonders nach Feuerüberfällen – häufige Stellungswechsel sind sehr effektiv um Gegner zu verwirren. \\

\subsubsection{Kommando:<<Nehmt ein!>>}

	Das bedeutet für den Trupp, dass eine gegnerische Stellung und/oder eines Gebiet (umkämpft oder nicht umkämpft) unter allen Umständen eingenommen und erobert werden soll. Einnehmen signalisiert dem Soldaten, dass Feindkontakt möglich ist. \\

\subsubsection{Kommando: <<Haltet …!>>}

	Das bedeutet nichts anderes – als „bleiben Sie IN der Stellung und verteidigen Sie die Stellung>>. \\

\subsubsection{Kommando: <<Melden bei Vollzug!>>}

	Der Soldat soll, sobald er seinen Auftrag erledigt hat, sich beim Auftraggeber melden – per Funk oder direkter Ansprache. \\

\subsubsection{Kommando: <<Status>>}

	Der Soldat gibt den Gesundheitsstatus und seinen Munitionsstatus, mittels Ampelsystem, durch. (Beispiel: << Hier 5, Grün, Gelb >>) Ggf. können weitere Meldungen kurz und knackig mit gegeben werden. (Beispiel: <<noch 500 Meter bis zu eurer Stellung >>) \\

\subsubsection{Feuerfreigaben}

	Es versteht sich von selbst, dass wir nicht auf alles zur jeder Zeit schießen können. Daher gibt es die sogenannten Feuerfreigaben, wann geschossen werden darf und wann nicht. Wir unterschieden 3 Typen: \\

\paragraph{<<Status - Feuer halten>>  oder auch <<Feuerstatus rot>> }

	Nur schießen, wenn es um das blanke Überleben geht. „Feuer halten“ wird ausgegeben, wenn wir leise und unerkannt in eine Stellung einsickern wollen. Feuergefechte sollten unter allen Umständen vermieden werden, da Schüsse die eigene Position verraten und Kameraden gefährden können. \\

\paragraph{<<Status - Feuer erwidern>>  oder auch <<Feuerstatus gelb>> }

	Bei erkanntem Feind, der uns noch nicht uns bekämpft (kalt), wird der Feind gemeldet und es wird gewartet, was der Truppführer entscheidet. Es darf bei Beschuss auf die eigene Stellung zurückgeschossen werden, wenn der Feind klar erkannt wurde (heiß). \\

\paragraph{<<Status - Feuer frei>> oder auch <<Feuerstatus grün>> }

	Jeder erkannte Feind darf bekämpft werden. Diese Freigabe wird beispielsweise bei einem Feuerüberfall gegeben, oder in einem Verteidigungsszenario aus einer befestigten Stellung heraus. \\

