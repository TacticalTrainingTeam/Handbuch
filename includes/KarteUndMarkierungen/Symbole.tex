\newpage

\subsection{Symbole}
	Die Vanille Symbole werden auf der Karte gesetzt in dem man, im Kartenmenü, per Doppelklick auf die gewünschte Position einen Punkt setzt. Standardmäßig kann mit den Cursor Pfeilen \begin{math} >>\uparrow<<, >>\downarrow<< \end{math} die Symbole durchgeschaltet werden. Mit >>Shift<< wird die Farbe geändert. Eine Rotation der Pfeile ist leider nicht vorgesehen. \\
	Die Vanille Kartenmarkierungen sollten nur im "Gruppen Channel" gesetzt werden.

\begin{longtable}{p{3cm} p{15cm}}
	\includegraphics[scale=1]{./Grafiken/KarteUndMarkierungen/HQ.png}			& 		Hauptquartier / Stützpunkt \\
	\includegraphics[scale=1]{./Grafiken/KarteUndMarkierungen/Punkt.png}			&		Punkt, grün Haus sauber, oder auch Wegpunkte \\
	\includegraphics[scale=1]{./Grafiken/KarteUndMarkierungen/Pfeil.png}			&		Pfeil / Richtung \\
	\includegraphics[scale=1]{./Grafiken/KarteUndMarkierungen/Start.png}			&		Startpunkt \\
	\includegraphics[scale=1]{./Grafiken/KarteUndMarkierungen/Endpunkt.png}		&		Endpunkt \\
	\includegraphics[scale=1]{./Grafiken/KarteUndMarkierungen/Treffpunkt.png} 		&		Treffpunkt \\
	\includegraphics[scale=1]{./Grafiken/KarteUndMarkierungen/Aufgabe.png}		&		Aufgabe / Ziel \\
	\includegraphics[scale=1]{./Grafiken/KarteUndMarkierungen/Angriffsrichtung.png}	& 		Angriffsrichtung \\
	\includegraphics[scale=1]{./Grafiken/KarteUndMarkierungen/Achtung.png}		&		Vorsicht / Warnung \\
	\includegraphics[scale=1]{./Grafiken/KarteUndMarkierungen/FrageUnbekannt.png}	& 		Unbekannt \\
\end{longtable}

\newpage

	Kartenmarkierungen auf dem Nato-tablet erfolgt durch ...

\begin{longtable}{p{3cm} p{15cm}}
	\includegraphics[scale=1]{./Grafiken/KarteUndMarkierungen/Unbekannt.png}		& 		Unbekannt \\
	\includegraphics[scale=1]{./Grafiken/KarteUndMarkierungen/Recon.png}		 	& 		Recon / Späher \\
	\includegraphics[scale=1]{./Grafiken/KarteUndMarkierungen/motorisierteInfanterie.png}		 & 		motorisierte Infanterie \\
	\includegraphics[scale=1]{./Grafiken/KarteUndMarkierungen/mechanisierteInfanterie.png}		 & 		mechanisierteInfanterie\\
	\includegraphics[scale=1]{./Grafiken/KarteUndMarkierungen/Panzer.png}		& 		Panzer \\
	\includegraphics[scale=1]{./Grafiken/KarteUndMarkierungen/Infanterie.png}		& 		Infanterie \\
	\includegraphics[scale=1]{./Grafiken/KarteUndMarkierungen/Flugzeug.png}		& 		Flugzeug \\
	\includegraphics[scale=1]{./Grafiken/KarteUndMarkierungen/Hubschrauber.png}	&	 	Hubschrauber \\
	\includegraphics[scale=1]{./Grafiken/KarteUndMarkierungen/Drohne.png}		& 		Drohne \\
\end{longtable}