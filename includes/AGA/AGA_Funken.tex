\newpage
\subsection{Richtig Funken und Kommunizieren im Trupp}

	Zum generellen Gebrauch des Funkgeräts sei auf den TFAR-Mod \autoref{TFAR} verwiesen. Weiterhin sollte man die eigene Stimmlautstärke nicht unbedingt auf \glqq Laut\grqq\, (\glqq yelling\grqq) stellen. Das dient dazu bei mehreren eng geführten Trupps nicht bis zum anderen Trupp rüber zu schreien.\\
	Grundsätzlich sollten alle wichtigen Meldungen dem Truppführer über die Trupp interne Funkfrequenz mitgeteilt werden.\\

\subsubsection{Kontakmeldungen}
	Kontaktmeldungen  sollten möglichst nach einem bestimmten Schema ablaufen.  Auch hier gilt im Gefecht wird nicht jede Kontaktmeldung perfekt sein. Dennoch sollte man sich an dieser Meldeform möglichst nah orientieren.
		\begin{itemize}
		\item <<Kontakt>> oder <<Freunde>>
		\item Anzahl 
		\item Was
		\item Wo (grobe Richtung, Landmarke, Landmarken ggf. verfeinern, genaue Grad zahl) 
		\item Entfernung
		\item  weitere Verfeinerung wie zum Beispiel Bewaffnung, Alarmzustand (heiß oder kalt)
	\end{itemize}

	Beispiele:
	<<Kontakt … 1 (feindlicher) \ac{SPZ} … auf 4 Uhr … Rechts über der Hütte … auf 95° … ca. 500m>> \\
	<<Kontakt … 3 Infanteristen … auf 1 Uhr … Kommen über den Hügel … bei 25° … ein MG … ca. 300m … kalt>> \\

\subsubsection{Funkgespräch unterbrechen}
	Um eine aktuell durchgeführte Funkmeldung oder Gespräch zu unterbrechen, um eine Meldung abzusetzen von höherer Priorität, oftmals Feindkontakt, hat sich ein kurzes <<Break, Break>> oder <<Check, Check>> als sinnvoll heraus gestellt. \\
	Beispiel: \\
	TF: <<Wir gehen folgendermaßen vor Bud…>>  \\
	TM: <<Break, Break>> … <<Kontakt feindlicher Trupp auf 2 Uhr  100m nähert sich>> …  \\
	TF: <<Verstanden, Feuer frei>> \\

\subsubsection{sonstige Meldungen}
	Bei betreten oder verlassen von Fahrzeugen wird sich kurz und prägnant gemeldet das man dieses entsprechend betreten oder verlassen hat, damit erhält der TF die Information wo sein Trupp sich befindet und die Truppmitglieder wissen wie sie sich zu verhalten haben. (Erst wenn 3 meldet das er sitzt kann 2 aufsitzen) \\
	Beispiel:  <<3, Sitzt>> \\

	Meldungen über aufgebaute Sicherung können sehr nützlich sein um dem TF über den Status der Sicherung zu empfehlen. \\
	Beispiel: << Hier 6, Sicherung nach 6 Uhr steht>> \\

	Das letzte Trupp Mitglied (üblw. die Nr.6) meldet den Status der Bewegung. \\
	Beispiel: \\
	TF: <<Im Stack Marsch>> \\
	6: <<Trupp in Bewegung>>  \\
	Statusmeldung und durchzählen. \\
	Nach heftigen, ggf. unübersichtlichen Feindbeschuss ist es sinnvoll den Status des Trupps abzufragen. Sollte der Truppführer nicht erreichbar sein kann auch ein Trupp Mitglied die Statusabfrage anordnen. \\
	Es wird generell von 1 an heruntergezählt. (Wenn 1 die Statusmeldung  anordnet entfällt die Meldung der 1 i.d.R.)  \\
	Aufbau der Meldung: \\
	<<Nummer, Gesundheitsstatus, ggf. Besonderheiten>>  \\
	Beispiel:\\
	 <<1 hier 5, Kommen>> … (keine Reaktion)  <<Hier 5, Trupp Status durchgeben>> … (kurz warten 1 und 2 melden sich nicht) ... <<Hier 3, Status Rot>> … <<Hier 4, Status Grün, zwei Verletzte auf dem Hügel>> … usw. \\

\subsubsection{Wichtigste Funkabkürzungen und Begriffe}
	\begin{tabular}{|p{2cm}|p{4cm}|p{3cm}|p{4cm}|} \hline
		SPZ &	Schützenpanzer	& Standby &	Bitte warten \\ \hline 
		KPZ	 & Kampfpanzer	& Kalt	& keine Gegner, Gegner haben uns nicht erkannt \\ \hline
		AA	& (Anti Air) Flugabwehr	& Heiß &	Gegner eröffnen Feuer, haben uns entdeckt \\ \hline
		AT	& (Anti Tank) Anti Panzer	& OPL	& Operationsleitung \\ \hline
		MG	& Maschinengewehr	& CAS	& Close Air Support \\ \hline
	\end{tabular}


