%TODO: Links, Erklärungen, Einleitung
\section{Das AGA-Handbuch}
\subsection{Über die Grundausbildung}
\subsection{Organisation der Trupps}
In diesem Abschnitt werden die einzelnen Strukturelemente und die Organisation des \ac{TTT} während einer typischen Mission erklärt. Dabei wird auf das Buddyteam, den Trupp, den Zug und die gesamte Struktur der Kräfte eingegangen.\\
Die Struktur lehnt sich dabei an das System der Bundeswehr sowie der Amerikanischen Streitkräfte an und wird im \ac{TTT} als 6 + 2 System so gelebt.\\
Ganz unten in der Hierarchie der Strukturelemente steht das Buddyteam, welches aus zwei Soldaten besteht. Sie bilden die kleinste, elementarste Einheit. Zwei oder mehr Buddyteams werden durch einen Truppführer koordiniert und bilden so einen Trupp. Zwei Trupps bilden wiederum einen Zug, der durch die Zugführung geleitet wird. Alle Züge und Elemente werden von den Anweisung seitens der Operativen Planung (\ac{OPL}) geleitet.  
\begin{longtable}{| >{\columncolor{backcolor}}l |p{3cm} | p{10cm} |} 
	\caption[Truppübersicht]{Übersicht der Truppen im TTT} \\
	\hline
	\textbf{Farbe} & \textbf{Symbol} & \textbf{Beschreibung} \\
	Gelb & \includegraphics [width=20mm]{./Grafiken/Abschnitt/TrGelb} & Kompanieführung -- koordiniert alle an Operationen beteiligte Truppenteile und liefert Feuerunterstützung durch Artillerie. \\
	\hline
	Gold & \includegraphics[width=20mm]{./Grafiken/Abschnitt/TrGold} & Fernspäher -- führt Aufklärung mit Drohnen und Scharfschützenteams und bekämpft Einzelziele mit hoher Priorität.\\
	\hline
	Grau & \includegraphics[width=20mm]{./Grafiken/Abschnitt/TrGrau} & Kommando Spezialkräfte -- Spezialkräfte zur gesonderten Verwendung, welche parallel zu den regulären Kräften operieren und deren Vorgehen durch Aufklärung, Sabotage und gezielte Bekämpfung von Einzelzielen unterstützt. \\
	\hline
	Grün &  & Zugführung -- befehligt Fallschirmjägertrupp Schwarz und Pioniertrupp Blau, übernimmt im Notfall medizinische Erstversorgung und koordiniert \ac{MedEvac}, \ac{CAS} und Logistik wenn benötigt. \\
	\hline
	Braun &   &Zugführung -- koordiniert Fallschirmjägertrupp Rot und Pioniertrupp Violett, übernimmt im Notfall medizinische Erstversorgung und koordiniert \ac{MedEvac}, \ac{CAS} und Logistik wenn benötigt. \\
	\hline
	Schwarz & \includegraphics[width=20mm]{./Grafiken/Abschnitt/TrSchwarz} &Fallschirmjägertrupp -- bildet mit leichten Maschinengewehren und Antifahrzeugbewaffnung das Rückgrat der Operationen im Zug Grün. \\
	\hline
	Blau & \includegraphics[width=20mm]{./Grafiken/Abschnitt/TrBlau} & Pioniertrupp -- liefert Trupp Schwarz mit MG entsprechende Unterstützung und übernimmt Räum- und Sprengaufgaben.\\
	\hline
	Rot & \includegraphics[width=20mm]{./Grafiken/Abschnitt/TrRot} & Fallschirmjägertrupp -- bildet mit leichten Maschinengewehren und Antifahrzeugbewaffnung das Rückgrat der Operationen im Zug Braun.\\
	\hline
	Violett &  & Pioniertrupp -- liefert Trupp Rot mit MG entsprechende Unterstützung und übernimmt Räum- und Sprengaufgaben \\
	\hline
	Weiß & \includegraphics[width=20mm]{./Grafiken/Abschnitt/TrWeiss} & \ac{MedEvac} -- Unterstützt Operationen mit Versorgungs- und Transportkapazität für Verwundete. \\
	\hline
	Silber & \includegraphics[width=20mm]{./Grafiken/Abschnitt/TrSilber} & \ac{CAS} und Logistik -- Stellt Fahrzeuge und Personal bereit, mit denen Transport, Logistik und Gefechtsunterstützung durchgeführt werden. \\
	\hline		
\end{longtable}
\newpage
\subsubsection{Organisation und Befehlshierarchie}
%TODO: Erklärung warum und wie es funktioniert
Das Ziel jeder militärischen Organisation ist es, die gesetzten Ziele schnell und effizient zu erreichen: wie kleine, fein abgestimmte Zahnräder sollen Befehlender und Befehlsgeber miteinander interagieren und auf rapide Veränderungen ebenso rapide reagieren. Dazu ist die Einhaltung der Befehlshierarchie, der Funk- und Meldediziplin absolut notwendig.
\\Im englischsprachigen Raum existiert der Begriff \textit{chain-of-command}, welcher im deutschen Sprachraum äquivalent mit den Begriffen \textit{Dienstweg} und, besonders im militärischen, \textit{Befehlshierarchie} ist. Man versteht darunter, wie und an wen Befehle ausgegeben und angenommen werden. Ganz wichtig ist: Sofern nicht explizit anders geregelt (Grün: "Schwarz meldet Sichtungen direkt an \ac{OPL}") werden alle Befehle nur an direkt Unterstellte ausgegeben und Meldungen nur an direkte Vorgesetzte gemacht. Als Beispiel sei genannt, dass die Zugführung einem Soldaten im Regelfall nicht einen expliziten Befehl geben ("Sicherungsbereich Nord-nord-ost") wird, dafür ist der Truppführer zuständig. Dies geschieht, um eine klare Übersicht und Kontrolle auch in komplizierten und unübersichtlichen Situationen zu haben - und fußt auf Vertrauen. Dem Vertrauen, dass der Befehlende nur solche Befehle ausgibt, die der Befehlsempfänger unter der Berücksichtigung seiner momentanen Lage (Ausrüstung, Position, Zustand, ...) realistischer Weise schaffen kann. Umgekehrt vertraut ein Befehlender darauf, dass ein solcher Befehl ausgeführt und der Ausgang gemeldet wird: Kommt es beispielsweise zu momentan unauflösbaren Schwierigkeiten, werden diese gemeldet.
\\Jeder sollte dabei im Hinterkopf behalten, dass...
\begin{itemize}
	\item ... ein schlechter Befehl manchmal besser ist als kein Befehl.
	\item ... mit wenigen Ausnahmen jeder Spieler auch ein Mensch ist. 
	\item ... mit Herz und Verstand befehligt und gehorcht wird. 
	\item ... für jede Diskussion die \ac{AAR} da ist.
	\item ... das Ziel ist, jederzeit immer besser zu werden.
	\item ... eine "Scheißegal"-Attitüde dazu führen muss, dass du den Server verlässt.
\end{itemize}
\subsubsection{Die Trupps}
%TODO: Ausführliche Beschreibung der Aufgabengebiete
\subsubsection{Uniformen und Zeichen}
In den meisten Fällen lassen sich Mitglieder des \ac{TTT} an den Zeichen auf ihren Uniformen erkennen und identifizieren:
\begin{itemize}
	\item Patch auf dem linken Oberarm in Truppfarbe und mit dem entsprechenden Logo
	\item Auf der Rückseite des Helms ist die Nummer des Soldaten im Trupp angebracht
	\item Auf jedem Handschuh befindet sich das Logo des \ac{TTT} in der entsprechenden Truppfarbe
\end{itemize}

\subsection{Dein Buddy und du}
%TODO: Wichtige Lektionen einfügen
\centerline{\textit{Dies ist mein Buddy. Davon gibt es viele, aber dieser gehört mir.}}
Die Aussage einer Erzieherin \textit{"So, und jetzt nimmt jeder die Person links von ihm an die Hand"} fasst am besten zusammen, für was ein Buddyteam steht: Man passt gegenseitig aufeinander auf, sodass man nicht verloren geht. Egal ob beim Tauchen, bei einer Ralley, im Flugzeug, beim Pair-Programming oder im Scharfschützenteam - die Stärke der Kleinsten aller Gruppen besteht darin, dass zwei Personen an einem Strang ziehen und sich dabei gegenseitig unterstützen können.
\\Das Buddyteam im \ac{TTT} ist viele Dinge: die kleinste organisatorische Einheit, das solideste Fundament jedes Trupps, die beste Versicherung eines einzelnen Spielers das Ende der Mission zu sehen. Buddyteams werden üblicherweise so gebildet, dass sie einer speziellen Funktion folgen können: Sei dies als Maschinengewehrcrew, Fahrzeugbekämpfung, Aufklärung, Führung oder jeder anderen denkbaren Aufgabe. Buddyteams haben auch Nachteile. Darunter ist beispielsweise, dass sie üblicherweise zufällig für eine Mission gebildet werden, die beiden Personen sich nicht notwendigerweise verstehen und ähnliches. Regelmäßiges Spielen und Trainingkann diese Probleme eindämmen, jedoch niemals ganz.
\\Das zentrale Element eines solchen Teams bist immer \textit{DU} -- Dein Streben muss es ständig sein, ein besserer Buddy zu werden. Nach der AAR nicht nur Vorgesetzte und erfahrenere Spieler, sondern auch den für die Mission zugeteilten Buddy zu fragen was man verbessern kann sollte zum guten Ton eines \ac{TTT}-Spielers gehören. Auch während der Mission gibt es ein paar Dinge, über die man sich jederzeit im Klaren sein sollte:
\begin{itemize}
	\item Position meines Buddys
	\item Zustand meines Buddys
	\item Ausrüstung, momentaner Munitionsstand meines Buddys
	\item Grobe Blickrichtung oder Sicherungsrichtung meines Buddys
	\item ...
\end{itemize}
Auch hier hilft nur eins: Erfahren, denken, üben. So hilft es beispielsweise nichts, einfach so loszustürmen um seinen Buddy zu bergen und dabei selbst schwer verwundet zu werden. 

\subsection{Der Trupp}
Fasst man mindestens zwei Buddyteams zusammen, so entsteht ein klassischer Kommando-Trupp wie man ihn beispielsweise für den Häuserkampf oder Spezialeinsätze benötigt. Wie auch beim Buddyteam entsteht dabei die Stärke eines Trupps dadurch, dass sich die einzelnen Teams gegenseitig unterstützen und sichern können. Je besser ein Trupp aufeinander eingespielt ist, umso effizienter wird er. Und gerade bei einem 4-Mann-Trupp ist ein hohes Maß an Übung und Einspielen erforderlich, da bereits der Ausfall eines einzelnen Soldaten den Trupp signifikant an der Erfüllung seines Auftrags hindern kann.
\\Ein Trupp aus drei Buddyteams ist nicht nur robuster, sondern auch taktisch vielfältiger einsetzbar: so kann beispielsweise ein Buddyteam unter der Sicherung der beiden anderen Teams vorrücken. Er ist der Standardtrupp, der im \ac{TTT} Verwendung findet. Wie in solcher Trupp aussehen kann ist nachfolgend aufgeführt:
\begin{itemize}
	\item Buddy-Team 1, “Führung” 
	\begin{itemize}	
		\item Nr.1: Truppführer -  Führt den Trupp durch klare, vorbildliche Führung von vorn
		\item Nr.2: Stellv. Truppführer - Unterstützt den Truppführer indem er Funk oder ähnliche Aufgaben übernimmt
	\end{itemize}
\end{itemize}
\begin{itemize}
	\item Buddy-Team 2, “Schwere Waffen” 
	\begin{itemize}
		\item Nr.3: MG-Assistent - Trägt zusätzliche Munition für das MG, hilft beim einschießen auf die Ziele, lenkt das MG, trägt Munition, unterstützt mit Nahsicherung
		\item Nr.4: MG-Schütze - Trägt das MG und führt den Feuerkampf
	\end{itemize}
\end{itemize}
\begin{itemize}
	\item Buddy-Team 3, “Spezialisten” 
	\begin{itemize} 
		\item Nr.5: Spezialist 1 - Übernimmt zusammen mit seinem Buddy Spezialaufgaben wie beispielsweise Grenadier, \ac{AA}/\ac{AT}, \ac{EOD}, Sanitäter, Designated Marksman
		\item Nr.6: Spezialist 2 - Übernimmt zusammen mit seinem Buddy Spezialaufgaben wie beispielsweise Grenadier, \ac{AA}/\ac{AT}, \ac{EOD}, Sanitäter, Designated Marksman
	\end{itemize}
\end{itemize}
Es ist nicht ungewöhnlich, dass man über seine Nummer im Trupp oder die Nummer des Buddyteams angesprochen wird. Es ist daher von Vorteil, sich mit Stift und Papier zu merken in welchem Buddyteam man ist und welche Nummer man selbst und die anderen Leute haben.