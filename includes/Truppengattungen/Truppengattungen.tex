\section{Truppengattungen}
	In diesem Abschnitt werden die einzelnen Strukturelemente und die Organisation des \ac{TTT} während einer typischen Mission erklärt. Dabei wird auf das Buddyteam, den Trupp, den Zug und die gesamte Struktur der Kräfte eingegangen.\\
	Die Struktur lehnt sich dabei an das System der Bundeswehr sowie der Amerikanischen Streitkräfte an und wird im \ac{TTT} als 6 + 2 System so gelebt.\\
	Ganz unten in der Hierarchie der Strukturelemente steht das Buddyteam, welches aus zwei Soldaten besteht. Sie bilden die kleinste, elementarste Einheit. Zwei oder mehr Buddyteams werden durch einen Truppführer koordiniert und bilden so einen Trupp. Zwei Trupps bilden wiederum einen Zug, der durch die Zugführung geleitet wird. Alle Züge und Elemente werden von den Anweisung seitens der Operativen Planung (\ac{OPL}) geleitet.  

\subsection{Operationsleitung}
\includegraphics[width=20mm]{./Grafiken/Abschnitt/TrGelb}\\
Kompanieführung -- Trupp Gelb -- koordiniert alle an Operationen beteiligte Truppenteile und liefert Feuerunterstützung durch Artillerie.
\section{Zugführung}
\includegraphics[width=20mm]{./img/truppenordnung/zugfuehrung/TrBraun}\quad\includegraphics[width=20mm]{./img/truppenordnung/zugfuehrung/TrGruen}\\
Zugführung -- Trupp Grün und Trupp Braun -- befehligt Fallschirmjägertrupp Schwarz und Pioniertrupp Blau, übernimmt im Notfall medizinische Erstversorgung und koordiniert \ac{MedEvac}, \ac{CAS} und Logistik wenn benötigt. 
\subsection{Infanterie}
\includegraphics[width=20mm]{./Grafiken/Abschnitt/TrSchwarz}\quad\includegraphics[width=20mm]{./Grafiken/Abschnitt/TrRot}\\
Schwarz und Rot -- Fallschirmjägertrupp -- bilden mit leichten Maschinengewehren und Antifahrzeugbewaffnung das Rückgrat der Operationen im Zug Grün (Schwarz) und Braun (Rot).\\

\includegraphics[width=20mm]{./Grafiken/Abschnitt/TrBlau} \quad \includegraphics[width=20mm]{./Grafiken/Abschnitt/TrViolett}\\
Blau und Violett -- Pioniertrupp -- liefert Trupp Schwarz / Rot mit MG entsprechende Unterstützung und übernimmt Räum- und Sprengaufgaben.
\subsection{Spezialkräfte}
\includegraphics[width=20mm]{./Grafiken/Abschnitt/TrGold}\\
Gold -- Fernspäher -- führt Aufklärung mit Drohnen und Scharfschützenteams und bekämpft Einzelziele mit hoher Priorität.\\\\
\includegraphics[width=20mm]{./Grafiken/Abschnitt/TrGrau}\\
Grau -- Kommando Spezialkräfte -- Spezialkräfte zur gesonderten Verwendung, welche parallel zu den regulären Kräften operieren und deren Vorgehen durch Aufklärung, Sabotage und gezielte Bekämpfung von Einzelzielen unterstützt.

\subsection{Logistik}
\includegraphics[width=20mm]{./Grafiken/Abschnitt/TrSilber}\\
Silber -- Bussard -- Stellt Fahrzeuge und Personal bereit, mit denen Transport und Logistik durchgeführt werden.
\subsection{MedEvac}
\includegraphics[width=20mm]{./Grafiken/Abschnitt/TrWeiss}\\
Weiß -- \acf{MedEvac} -- Unterstützt Operationen mit Versorgungs- und Transportkapazität für Verwundete.
\subsection{Close Air Support}
\includegraphics[width=20mm]{./Grafiken/Abschnitt/TrSilber}\\
Silber -- Adler -- Stellt Fahrzeuge und Personal bereit,  Gefechtsunterstützung (\ac{CAS}) und Geleitschutz durchgeführt werden.
\section{Mechanisierte Infanterie}
Das Konzept der mechanisierten Infanterie befindet sich im Moment noch in Arbeit und wird in einer späteren Version des Handbuches hinzugefügt.
\section{Kampfpanzer}
Das Konzept der Kampfpanzer befindet sich im Moment noch in Arbeit und wird in einer späteren Version des Handbuches hinzugefügt.
\section{Artillerie}
Das Konzept der Artillerie befindet sich im Moment noch in Arbeit und wird in einer späteren Version des Handbuches hinzugefügt.