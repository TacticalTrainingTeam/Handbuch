\pagebreak
\section{Verhalten bei Feindkontakt}
	Wird ein Feind ausgemacht, so meldet ein Truppmitglied diesen an den TF.
	\par\medskip
	Dies könnte im Truppfunk so gemeldet werden:
	\begin{hint}
		1 hier 6, Kontakt, stationäres MG auf 290°, Entfernung ca. 800\,m, Kalt
	\end{hint}
	Eine andere Situation ergibt sich wenn der Feind das Feuer auf die eigene Stellung eröffnet. Hier hängt es von der Ausgabe des Feuerstatus ab. Ist Feuerstatus rot ausgegeben wird das Feuer nur erwidert wenn es ums blanke überleben geht.
	\par\medskip
	Hier kommen die Kommandos >>Stellung wechseln<<, >>Feuer Frei<<, >>Ausweichen<< oder >>Verzögern und Ausweichen<< in Frage.\footnote{Ausführlich in \nameref{sec:kommando} bzw. \nameref{sec:bewegung_gelaende} beschrieben}
	\par\medskip
	In allen Fällen gibt der TF das Kommando aus, wie sich verhalten werden soll.