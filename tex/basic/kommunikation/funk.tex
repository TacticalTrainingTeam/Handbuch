	Zum generellen Gebrauch des Funkgeräts sei auf den TFAR-Mod verwiesen. Weiterhin sollte die eigene Stimmlautstärke nicht auf >>Laut<<, (>>yelling<<) stellen. Dies verhindert, dass die Kommunikation, mehreren eng geführten Trupps, vermischt wird.\\
	Grundsätzlich sollten alle wichtigen Meldungen dem Truppführer, über die Trupp interne Funkfrequenz, mitgeteilt werden.
 
\subsection{Kontakmeldungen}
	Kontaktmeldungen sollten im Optimalfall nach einem bestimmten Schema ablaufen. Hierbei gilt im Gefecht wird nicht jede Kontaktmeldung perfekt sein. Dennoch sollte die Meldung möglichst nach diesen Schema erfolgen.
	\begin{itemize}
		\setlength{\itemsep}{-4pt}
		\itemsep-4pt
		\item >>Kontakt<< oder >>Freunde<<
		\item Anzahl 
		\item Was
		\item Wo (Grobe Richtung, Landmarke (ggf. verfeinern), Gradzahl auf Kompass) 
		\item Entfernung
		\item Weitere Angaben (Bewaffnung, Alarmzustand (heiß oder kalt), Bewegungsrichtung, ...)
	\end{itemize}

	Beispiele:
	\begin{longtable}{P{0.95\linewidth}}
	\toprule
	Kontakt -- 1 feindlicher \acs{SPZ} -- auf 4 Uhr -- Rechts über der Hütte -- auf 95° -- ca. 500m\\
	\rcg Kontakt -- 3 Infanteristen -- auf 1 Uhr -- Kommen über den Hügel -- bei 25° -- ein \acs{MG} -- ca. 300m -- kalt\\
	\bottomrule
	\end{longtable}	
\subsection{Funkgespräch unterbrechen}
	In bestimmten Situationen ist es erforderlich eine aktuelle Funkmeldung oder ein Gespräch zu unterbrechen, um eine Meldung abzusetzen von höherer Priorität, wie z.\,B. Feindkontakt zu melden. Dabei stellt >>Break, Break<< eine kurze und prägnante Aussage dar.\\
	Beispiel:
	\begin{longtable}{P{0.95\linewidth}}
	\toprule
	TF: >>Wir gehen folgendermaßen vor Bud\dots<<\\
	\rcg TM: >>Break, Break<< \dots >>Kontakt feindlicher Trupp auf 2 Uhr 100\,m nähert sich<<\\
	TF: >>Verstanden, Feuer frei!<<\\
	\bottomrule
	\end{longtable}		

\subsection{Sonstige Meldungen}
	\paragraph*{Betreten von Fahrzeugen}
	\label{para:fahrzeug-betreten}
	Bei betreten oder verlassen von Fahrzeugen wird kurz und prägnant gemeldet, dass der Soldat das entsprechend Betreten betreten oder verlassen hat. Damit erhält der TF die Information wo sein Trupp sich befindet und die Truppmitglieder wissen wie sie sich zu verhalten haben. (Erst wenn 3 meldet, dass er sitzt kann 2 aufsitzen)\\
	Beispiel: >>3, Sitzt<<

	\paragraph*{Sicherungsposition melden}
	Meldungen über aufgebaute Sicherung sind nützlich Details, die den TF über den Status der Sicherung informieren. \\
	Beispiel: >>Hier 6, Sicherung nach 6 Uhr steht<<

	\paragraph*{Trupp in Bewegung}
	Das letzte Truppmitglied (üblw. die Nr.\,6) meldet den Status der Bewegung.\par
	Beispiel:
	\begin{longtable}{P{0.95\linewidth}}
	\toprule
	TF: >>Trupp Marsch<<\\
	\rcg TM 6: >>Trupp in Bewegung<<\\
	\bottomrule
	\end{longtable}		
	
	\paragraph*{Statusmeldung und durchzählen}
	\label{sec:Status}
	Nach heftigen, ggf. unübersichtlichen Feindbeschuss ist es sinnvoll den Status des Trupps abzufragen. Die Meldung gibt den Gesundheitsstatus und Munitionsstatus, mittels Ampelsystem durch. Sollte der Truppführer nicht erreichbar sein kann auch ein Trupp Mitglied die Statusabfrage anordnen.\par
	Es wird generell von 1 an heruntergezählt. (Wenn 1 die Statusmeldung anordnet entfällt die Meldung der 1 i.\,d.\,R.)  \\
	Aufbau der Meldung: >>Nummer, Gesundheitsstatus, ggf. Besonderheiten<<
	\par\medskip
	Beispiel:
	\begin{longtable}{P{0.95\linewidth}}
	\toprule
	>>1 hier 5, Kommen<< \dots (keine Reaktion)\\  
	>>Hier 5, Trupp Status durchgeben<< \dots (kurz warten 1 und 2 melden sich nicht)\\ 
	\rcg >>Hier 3, Status Rot, Gelb<< \dots\\ 
	\rcgg >>Hier 4, Status Grün, Grün, zwei Verletzte auf dem Hügel<< \dots usw.\\
	\bottomrule
	\end{longtable}		

\subsection{Wichtigste Funkabkürzungen und Begriffe}
	\begin{longtable}{p{0.1\linewidth}p{0.25\linewidth}p{0.05\linewidth}p{0.1\linewidth}p{0.35\linewidth}} 
		\toprule
		\acs{SPZ}	& \acl{SPZ}	&& Standby	& Bitte warten \hfil\\ 
		\midrule
		\acs{KPZ}	& \acl{KPZ}	&& Kalt		& keine Gegner; Gegner haben uns nicht erkannt\\ 
		\acs{AA}	& \acl{AA}	&& Heiß 		& Gegner eröffnen Feuer, haben uns entdeckt \\ 
		\acs{AT}	& \acl{AT}	&& \acs{OPL}	& \acl{OPL} \\ 
		\acs{MG}	& \acl{MG}	&& \acs{CAS}	& \acl{CAS} \\ 
		\bottomrule
	\end{longtable}


