\subsection{Kommandos -- die wichtigsten Befehle}
\label{sec:kommando}
\medskip %TODO Kurzer Satz
\paragraph*{Kommando: >>Deckung!<<}
	Die Deckung ist ein situationsbedingter Aufenthaltsort.	\par
	Wird man vom Feind überrascht, ist der Soldat angewiesen, sofort Deckung im nahen Umfeld (Radius 20 Meter) zu suchen – ohne die Trupp-Struktur aufzulösen. Dabei gilt zu beachten, dass die Buddy-Teams immer im Verbund bleiben.\par
	Als Deckung zählt alles, was als beschusssicher gilt: Häuser, Felsen, Mauern, Senken. Ist keine Deckung in Laufweite, wird sofort Bodenlage eingenommen und auf den Feind ausgerichtet. Bei indirektem Beschuss (Granaten, etc.) wird ein besonders weiter Abstand zwischen den Kameraden gesucht.
	\par\medskip
	Der Befehl vom Truppführer lautet >>Deckung!<< oder als ausführlicheres Beispiel >>Deckung, 6 Uhr, hinter der Mauer!<<

\paragraph*{Kommando: <<Bezieht Stellung bei \dots!>>}
	Die Stellung ist ein selbst gewählter Aufenthaltsort.\par
	Die Stellung ist idealerweise ein gewählter und geschützter Aufenthaltsort (Haus, Senke, Felsen, Mauer) und lässt sich leicht verteidigen. Die Soldaten haben dabei maximale Deckung eingenommen und sichern.\par
	Der Truppführer gibt den Befehl aus, um einen Aufenthaltsort festzulegen, wo sich der Trupp aufhalten soll. Daraus folgt automatisch, dass der Trupp den Ort erreicht und vorläufig sichert. Der Sicherungsschwerpunkt liegt automatisch auf vermutete Feindstellungen. Mit dem Zusatz <<gedeckt>> betont er, dass der Trupp eine Aufklärung durch den Feind unter allen Umständen vermeiden soll. In der Praxis: Nähern in die gedeckte Stellung unter Sicht- und Deckungsschutz. Dort in die Stellung getarnt beziehen und halten. In der Stellung erfolgt vom Truppführer:

\paragraph*{Kommando: >>Bekämpfen \dots!<<}
	Der Soldat soll einen erkannten Feind bekämpfen. Hierbei ist es wichtig, genau abzuwägen, wie viel Einweisung benötigt wird, damit das Kommando erfolgreich ausgeführt werden kann. Der Soldat meldet bei erfolgreicher Absolvierung des Auftrags >>Gegner, YXZ, bekämpft!<<

\paragraph*{Kommando: >>Feuer frei!<<}
	Das ist die direkte Anweisung, JETZT zu schießen. Kann ergänzt werden mit >>Feuern, 3 Salven<< oder sonstigen Einschränkungen.

\paragraph*{Kommando: >>Unterdrückungsfeuer!<<}
	Das ist die direkte Anweisung, in die generelle Richtung des Gegners zu schießen, um ihn in die Deckung zu zwingen und ihn an Gegenfeuer oder Manövern zu hindern.

\paragraph*{Kommando >>Gezieltes Feuer!<<}
	Das ist die direkte Anweisung, den Gegner mit gezielten Salven oder Einzelschüssen direkt zu treffen -- also das Gegenteil vom Unterdrückungsfeuer. In der Regel wird bei der Feuerfreigabe erwartet, dass gezieltes Feuer abgegeben wird.

\paragraph*{Kommando >>Feuerfreigabe erteilt<<}
	Grundsätzliche Freigabe den Feuerkampf eigenständig zu eröffnen. Das heißt, sobald Gegner in Sicht sind und der Feuerkampf ist aussichtsreich, darf geschossen werden. Dies gilt bis zum Widerruf.

\paragraph*{Kommando >>Feuer nur auf Freigabe<<}
	Vor dem Schuss muss Feuerfreigabe angefordert werden.

\paragraph*{Kommando: >>Stopfen!<<}
	Der Soldat stellt sofort das >>Schießen<< ein.

\paragraph*{Kommando: >>Bezieht Stellung bei \dots!<<}
	Die Stellung ist ein selbst gewählter Aufenthaltsort. Idealerweise ist diese ein geschützter Aufenthaltsort (Haus, Senke, Felsen, Mauer) und lässt sich leicht verteidigen. Die Soldaten haben dabei maximale Deckung eingenommen und sichern.\par
	Der Truppführer gibt den Befehl aus, um einen Aufenthaltsort festzulegen, wo sich der Trupp aufhalten soll. Daraus folgt automatisch, dass der Trupp den Ort erreicht und vorläufig sichert. Der Sicherungsschwerpunkt liegt automatisch auf vermutete Feindstellungen. Mit dem Zusatz >>gedeckt<< betont er, dass der Trupp eine Aufklärung durch den Feind unter allen Umständen vermeiden soll. In der Praxis: Nähern in die gedeckte Stellung unter Sicht- und Deckungsschutz. Dort in die Stellung getarnt beziehen und halten. In der Stellung erfolgt vom Truppführer:
		\begin{itemize}
			\item Kontaktaufnahme mit dem Zugführer 
			\item Geländetaufe 
			\item Verteilung von Feuerbereichen 
			\item Vorausschauende Planung (Wegplanung, Korridore, weitere Stellungen) 
			\item Kommando: <<Bezieht gedeckte Stellung bei \dots!>> 
		\end{itemize}

\paragraph*{Kommando: >>Wechselt die Stellung!<<}
	Der taktische Stellungswechsel ist ein eigenständiger Positionswechsel in eine andere Stellung -- beispielsweise, weil man befürchtet, die Stellung ist vom Feind bereits aufgeklärt wurde und damit Feindfeuer erwartet. Das gilt besonders nach Feuerüberfällen -- häufige Stellungswechsel sind sehr effektiv um Gegner zu verwirren.

\paragraph*{Kommando: >>Nehmt ein!<<}
	Das bedeutet für den Trupp, dass eine gegnerische Stellung und\,/\,oder ein Gebiet (umkämpft oder nicht umkämpft) unter allen Umständen eingenommen und erobert werden soll. Einnehmen signalisiert dem Soldaten, dass Feindkontakt möglich ist.

\paragraph*{Kommando: >>Haltet \dots!<<}
	Das bedeutet nichts anderes als IN der Stellung zu verbleiben und diese zu verteidigen.

\paragraph*{Kommando: >>Melden bei Vollzug!<<}
	Der Soldat soll, sobald er seinen Auftrag erledigt hat, sich beim Auftraggeber melden. Dies kann per Funk oder direkter Ansprache erfolgen.

\paragraph*{Kommando: >>Status<<}
	Der Soldat gibt den Gesundheitsstatus und seinen Munitionsstatus durch. Dieses Kommando deckt sich mit den Abschnitt \ref{sec:Status}.

\subsubsection{Feuerfreigaben}
	Es versteht sich von selbst, dass wir nicht auf alles zur jeder Zeit schießen können. Daher gibt es die sogenannten Feuerfreigaben, wann geschossen werden darf und wann nicht. Wir unterschieden 3 Typen:

\paragraph*{>>Feuerstatus rot<<}
	Alternativ: >>Status -- Feuer halten<<\par
	Nur schießen, wenn es um das blanke Überleben geht. >>Feuer halten<< wird ausgegeben, wenn wir leise und unerkannt in eine Stellung einsickern wollen. Feuergefechte sollten unter allen Umständen vermieden werden, da Schüsse die eigene Position verraten und Kameraden gefährden können.
\paragraph*{>>Feuerstatus gelb<<}
	Alternativ: >>Status -- Feuer erwidern<<\par
	Bei erkanntem Feind, der uns noch nicht uns bekämpft (kalt), wird der Feind gemeldet und es wird gewartet, was der Truppführer entscheidet. Es darf bei Beschuss auf die eigene Stellung zurückgeschossen werden, wenn der Feind klar erkannt wurde (heiß).
\paragraph*{>>Feuerstatus grün<<}
	Alternativ: >>Status -- Feuer frei<<\par
	Jeder erkannte Feind darf bekämpft werden. Diese Freigabe wird beispielsweise bei einem Feuerüberfall gegeben, oder in einem Verteidigungsszenario aus einer befestigten Stellung heraus.
%
\subsubsection{Zusammenfassung Kommandos}
	\begin{longtable}{p{0.3\linewidth}p{0.65\linewidth}} 
		\toprule
		\textbf{Befehl} & \textbf{Bedeutung}\\
		\midrule
		Deckung! & Beschusssichere Stellung in 20\,m Umkreis suchen mit Buddy, ansonsten auf den Boden legen.\\
		Bekämpfen! & Erkannten Feind bekämpfen, bei Vollzug mit >>Bekämpft<<, bestätigen.\\
		Feuer Frei! & Feuer sofort eröffnen.\\
		Unterdrückungsfeuer! & Durch gezielte Schüsse in Feindrichtung diesen am kämpfen oder bewegen hindern.\\
		Gezieltes Feuer! & Gegner mit gezielten Salven oder Einzelschüssen bekämpfen. Wird bei \glqq Feuer frei\grqq\, vorausgesetzt.\\
		Feuerfreigabe erteilt! & Ist Feuerkampf aussichtsreich, darf dieser begonnen werden.\\
		Feuer nur auf Freigabe! & Feuerkampf ist nur auf expliziten Befehl hin zu beginnen.\\
		Stopfen! & Das Feuern ist \textbf{sofort} einzustellen.\\
		Bezieht Stellung bei \dots! & Mit Trupp oder Buddy zu einer Position gehen und entsprechende Sicherung aufbauen.\\
		Wechselt die Stellung! & Mit Trupp oder Buddy oder alleine in andere Stellung verlegen.\\
		Nehmt \dots ein! & Das angegebene Ziel ist unter allen Umständen zu erobern und zu sichern.\\
		Haltet! & In der Stellung bleiben und diese verteidigen.\\
		Melden bei Vollzug! & Rückmeldung bei Abschluss eines Befehls oder Auftrags.\\
		Status! & Gesundheitsstatus und Munitionsstatus mit Ampelsystem durchgeben.\\
		\bottomrule
	\end{longtable}
