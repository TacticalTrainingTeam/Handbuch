\pagebreak
\section{Umgang mit Fahrzeugen}
\subsection{Auf- und Absitzten}
Fahrzeuge (sowohl Boden-, als auch Luftfahrezeuge) werden in umgekehrte Trupp Reihenfolge betreten. (Siehe auch \nameref{para:fahrzeug-betreten})
\par\medskip
\begin{hint}
	Kommando >>Verlassen<<: Alle Truppteile verlassen das Fahrzeug
\end{hint}
\begin{hint}
	Kommando >>Absitzen<<: Schütze bleibt, Fahrer bleibt im Fahrzeug bzw. in der Nähe des Fahrzeuges, restliche Personen verlassen das Fahrzeug und bauen Sicherung auf.
\end{hint}

\subsection{Sicherungsbereich}
Der Sicherungsbereich nach Verlassen des Fahrzeuges entspricht den Standsicherungsbereichen für  Trupps (siehe \nameref{subsec:360sicherung}, \autoref{subsec:360sicherung}). Mehrere Trupps können sich z.\,B. 180° Sicherungsbereiche abstimmen. 
\par\medskip
Wichtig:
\begin{itemize}
	\item Bewegungskorridor vor und hinter dem Fahrzeug immer frei lassen\\ 12 Uhr ist immer die Fahrzeugfront
	\item Ein Fahrzeug in ARMA ist keine Deckung, wenn der Gegner AT-Waffen besitzt
	\item Abstand halten:\\ Wenn nicht anders befohlen, 20 Meter Abstand vom Fahrzeug\\ Fahrzeuge tendieren zum Explodieren, bieten schlechte Deckung und brauchen zudem jederzeit Platz für Manöver
\end{itemize}
\begin{figure}[htbp]
	\centering
	\includegraphics[width=15cm]{../img/basic/fahrzeug/fahrzeug_verlassen}
	\caption{Absitzen und Sicherung beim Fahrzeug}
\end{figure}

