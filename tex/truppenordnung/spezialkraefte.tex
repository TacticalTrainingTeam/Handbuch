\section{Spezialtrupp}
\includegraphics[width=20mm]{./img/truppenordnung/spezialeinheiten/sf1}\quad
\includegraphics[width=20mm]{./img/truppenordnung/spezialeinheiten/sf2}\\
Spezialtruppen sind infanteristische Einheiten bestehend aus zwei bis sechs Mann mit einem klaren Aufgabenschwerpunkt -- dies kann vom klassischen Zwei"=Mann"=Scharfschützenteam bis zum Sechs-Mann-Kampftauchertrupp gehen. Sie sind die flexibelsten Einheiten innerhalb des TTTs und können entweder autark arbeiten oder im Verbund mit einem anderen Trupp oder Zug. Pro Mission existieren maximal zwei autark operierende Spezialtruppen, im Verbund mit einem Zug maximal einer.\\
Kommunikation erfolgt über Long"=Range, beim Arbeiten im Verbund zusätzlich über die Additional"=Short"=Range (Zugfunk). Kämpfende Einheiten wie z.B. Kommandotrupps oder Kampftaucher, in denen der Truppführer viel Mikromanagement leisten und der Trupp in direkte Feuergefechte verwickelt wird, benötigen zwingend einen separaten Funker. In unterstützenden Einheiten wie z.B. Aufklärungsteams oder Mörserteams, die voraussichtlich nicht in direkte Feuergefechte verwickelt werden, kann (muss aber nicht) der Truppführer die Long"=Range"=Kommunikation mit übernehmen.\\
Mögliche Aufgabenschwerpunkte eines Spezialtrupps sind z.B.:

\begin{itemize}
	\setlength\itemsep{0em}
	\item JTAC-Team
	\item Aufklärungsteam (UAV)
	\item Autonome Kampfeinheit (UGV)
	\item Mörserteam
	\item schwere Feuerunterstützung (Schweres Maschinengewehr (HMG) / Granatmaschinengewehr (GMG))
	\item (schwere) Panzerabwehr / Flugabwehr (falls nicht bereits im Zug vorhanden)
	\item Pionier-Team (falls nicht bereits im Zug vorhanden)
	\item medizinische Versorgung/Unterstützung (falls kein MedEvac in der Mission vorhanden)
	\item Kommandokräfte (Infiltration)
	\item Scharfschützenteam
	\item Kampftaucher
\end{itemize}
Bei entsprechenden Rollen (schwere Waffen, Mörser, etc.) ist auf das Vorhandensein eines entsprechenden Assistenten zu achten.