\chapter{Führen}
\section{Der Befehle}
\label{subsec:befehl}
\begin{quote}
	\glqq Richtig ist besser\grqq
\end{quote}
Befehle sind laut Lehrbuch eine „Anweisung zu einem bestimmten Verhalten, die ein Vorgesetzter einem Untergebenen mit dem Anspruch auch Gehorsam erteilt“. Bei uns einfach „Mach dies und das und möglichst zum Zeitpunkt X“. Der Befehl ist in der Regel ein Gespräch bei uns zwischen Befehlsgeber und Befehlsempfänger. Der Befehl enthält einen Auftrag.\\
Der Auftrag ist das Kernstück und enthält die Informationen, auf die es ankommt.

\paragraph*{Befehle sind verbindlich!}\hfil\\
Befehle des „Vorgesetzten“ sind während der Mission umzusetzen, außer sie basieren auf fataler Fehleinschätzung aufgrund mangelnder Informationen, die dem Vorgesetzten nicht vorliegen. Dann gilt: Vorgesetzen informieren (beispielsweise über Feindkräfte, Lage, mangelnder Ausrüstung) und neue Befehlsausgabe abwarten. Absolut zu vermeiden sind Diskussionen. Dafür haben wir eine Nachbesprechung, wo alle Fehler und Fehlentscheidungen besprochen werden können. Weiterhin hat eine schlichte Befehlsverweigerung gewisse Konsequenzen.

\paragraph*{Befehle sind kurz und präzise!}\hfil\\
Richtig Befehle auszugeben ist für uns Spieler nicht einfach. 
Neben den vielen „öhm, ehm“ kommt besonders über Funk gerne mal kompletter Unsinn aus dem eigenen Mund. 
Weil wir erst während dem sprechen unsere Gedanken sortieren. 
Daher gilt „erste denken, dann sprechen“ und für Funksprüche „erst denken, drücken, sprechen“. Lieber ein paar Sekunden warten, konzentrieren, sortieren und dann die Befehlsausgabe.
Im folgenden bekommt wird der Aufbau des Befehls erläutert. Regel Nr.1: Konzentriert euch auf die Information, lasst den Schnickes und blümerante Dinge weg.
Faustregel zur Länge: Bleibt kurz und präzise. Floskeln weglassen -- und lieber später mal den Kameraden für die gute Umsetzung loben. 

\paragraph*{Möglicher Aufbau eines Befehls}
\begin{enumerate}
	\item Vorwarnung
	\item Rückmeldung
	\item Kurze Beschreibung der Situation (optional)
	\item Auftrag
	\item Auftragsbestätigung
	\item Auftragsauslösung
\end{enumerate}

Beispiel eines Befehls:
\begin{enumerate}
	\item ZF: >>Trupp Rot, bereit machen, neuer Auftrag in 1 Minute!<<
	\item TF: >>Trupp Rot bereit!<<
	\item ZF: >>Trupp Rot, aufgepasst: Lage und Vorhaben: Wir müssen im Dorf Kabalis eine Stellung errichten, dazu wollen wir ein Gebäude einnehmen. Trupp Blau wird den Vorstoß von der jetzigen Stellung hier absichern. Trupp Rot wird das Haus einnehmen. Wir erwarten nur leichten Widerstand -- evtl. ein paar verstreute Infanteristen in der Umgebung. So verstanden?<<
	\item TF: >>So verstanden!<<
	\item ZF: >>Trupp Rot, Euer Auftrag: Einnehmen des weißen, flachen Gebäudes mit rotem Dach auf 320 Grad, 200 Meter entfernt! Verstanden?<<
	\item TF: >>Verstanden, Einnehmen des weißen, flachen Gebäudes mit rotem Dach auf 320 Grad, 200 Meter entfernt.<<
	\item ZF: >>Gut, ausführen wenn bereit!<< (Truppführer entscheidet Zeitpunkt)\\ 
	ZF: >>Gut, sofort ausführen!/ Ausführen bei Erreichen des Punktes X, Ausführen bei 0815 Ortszeit<< (Zugführer entscheidet, wann und wo).
	\item ZF: >>Meldung bei Vollzug!<<
\end{enumerate}

\section{Der Auftrag}
\label{subsec:auftrag}
Beispielhafter Aufbau eines Auftrags:\\
In oben genanntem Beispiel ist folgender Auftrag zu finden:
>>Trupp Rot, Euer Auftrag: Einnehmen des weißen, flachen Gebäudes mit rotem Dach auf 320 Grad, 200 Meter entfernt! Verstanden?<<

Der Auftrag kommt mit dem Kommando >>Einnehmen<< (siehe \ref{sec:kommando}). Nach dem Kommando kommt das Ziel, welches eingenommen werden soll. Danach noch Richtung und Entfernung, damit der Auftragsempfänger auch das einnimmt, was der Auftragsgeber meint. Das Ganze übersichtlich dargestellt:

\begin{enumerate}
	\item Kommando
	\item Ziel
	\item Ortsangabe (Richtung \& Entfernung)
\end{enumerate}

\subsection{Auftragstaktik}\hfil
\begin{quote}
	\glqq Auftragstaktik Baby! Im \ac{TTT} arbeiten wir mit Aufträgen -- nicht mit der Hundeleine.\grqq
\end{quote}
Gute „Führer“ arbeiten IMMER auftragsbasiert -- also eine dezentrale Befehlsausgabe mit Vertrauen, dass der Auftragsempfänger den Auftrag selbständig im Sinne des Befehls erledigt. Man gibt als einen Auftrag – aber NICHT wie dieser Auftrag erreicht wird. Es kann dem Zugführer in der Regel völlig egal sein, über welchen Eingang das Haus genommen wird. Auftrag erteilen und alles Weitere dem Truppführer überlassen, er kann das am besten entscheiden, wie er seine Leute sicher und erfolgreich ins Haus bekommt.\\

Das Gleiche gilt für Truppführer: Hört auf mit dem Mikromanagement! Wenn ihr Eurem Trupp das Kommando gebt >>Deckung<< dann muss nicht gesagt werden, hinter welchen Felsen das jetzt sein soll. Dass gleiche gilt für „Stellung“ – der Trupp wird automatisch Stellung beziehen und bei Erfolg den Vollzug durchgeben. Wenn ein Truppführer einen Auftrag erteilt, beispielsweise ein Kontakt zu bekämpfen, dann geht das folgendermaßen:\\
\begin{tabular}{ll}
	& TF: >>5-6, 2 Feindkräfte auf 120 Grad, 200 Meter, am Spitzbaum, gesichtet?<<\\
	& 5-6: >>Gesichtet!<<\\
	& TF: >>5-6, Stellungswechsel, dann bekämpfen, Meldung bei Vollzug!<<\\
	& TF: >>Feuer frei!<< (Feuerfreigabe)\\
\end{tabular}


5 und 6 haben ihren Auftrag: Sie suchen sich eine geeignete Stellung mit genügend Deckung, überprüfen kurz die Flanken (rechts und links), sprechen sich kurz ab, legen Feuermodus fest und feuern nach Absprache untereinander. Danach melden Sie dem TF per Funk das erfolgreiche bekämpfen des Gegners. Anschließend verlegen sie zurück in Ausgangsstellung. 
\section{Führen als Operationsleitung}
\section{Führen als Zugführer}
\section{Führen als Truppführer}