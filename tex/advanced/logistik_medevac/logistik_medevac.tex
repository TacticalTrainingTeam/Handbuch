\section{Logistik und medizinische Unterstützung}
Wie anfänglich erwähnt, umfasst das Aufgabenfeld des FAC die Kommunikation und
Koordination verbündeter Lufteinheiten. Neben dem CAS, welcher das Kernthema dieser
SGA darstellt, beschäftigt der FAC sich im TTT auch mit der logistischen und
medizinischen Versorgung der Bodentruppen. Dabei wird für die medizinische Versorgung,
welche meist in Form einer Evakuierung auftritt der Fachbegriff MEDicalEVACuation
verwendet. Bei logistischer Unterstützung handelt es sich meist um die Lieferung von
Munition, sonstigen Materialen oder das Verlegen von Einheiten. Häufig werden im TTT
für beide Aufgabenbereiche Helikopter verwendet.

\subsection{Der 5-Liner im Detail}
Das TTT wendet für beide Versorgungsarten ein fünfzeiliges Informationsprotokoll an,
welches sich 5-Liner nennt. Folgende Informationen sind dabei enthalten.
\par\medskip
5-Liner:
\begin{enumerate}
	\item LZ 
	\item Heading
	\item Receiver
	\item Supply
	\item Type Mark
\end{enumerate}

Hier wird noch einmal die Bedeutung der einzelnen Lines im Detail erklärt.
\paragraph*{LZ}
Die LZ ist die Landezone, welche für das versorgende Flugobjekt eingerichtet wird. Der
Fokus liegt dabei auf der Funktionalität und Sicherheit des ausgewählten Gebietes.
\paragraph*{Heading}
Der Heading ist die Anflugrichtung, aus der sich das Flugobjekt der LZ nähert. Da im
Gegenteil zum CAS kein IP existiert, wird hier mit oberflächlichen Angaben, d.h.
Himmelsrichtungen gearbeitet.
\paragraph*{Receiver}
Der Receiver ist der Empfänger der Versorgung, dabei werden Truppfarben verwendet.
Der Empfänger ist ebenfalls für die Sicherung und Einrichtung der Landezone zuständig.
\paragraph*{Supply}
Supply beschreibt die Art der Versorgung. Im Falle von Materialien wie z. B. Munition wird
erst der Materialtyp und anschließend die benötigte Anzahl genannt. Sollten mehrere
Güter benötigt werden, wird nach Abschluss des ersten Materials erneut Typ und Anzahl
des zweiten Materials genannt. Hier noch einmal im Muster verdeutlicht.
\begin{hint}
\,[Material A], [Anzahl Material A] sowie [Material B], [Anzahl Material B] etc.
\end{hint}
Sollte es sich bei einem logistischen Auftrag um Verlegung von Truppen handeln, wird
nach folgendem Muster verfahren.
\begin{hint}
Verlegung von [Truppname] nach [Zielort (LZ)]
\end{hint}
Bei einer medizinischen Evakuierung wird der Medevac über die Art des Einsatzes, die
Anzahl der Verwundeten sowie ihren Status informiert. Auch hierzu ein Musterbeispiel.
\begin{hint}
Medevac, 1 Verwundeter, Status gelb.
\end{hint}
Sollte es sich um mehrere Verletzte handeln, wird der Status nicht einzeln angegeben,
sondern verallgemeinert.
\begin{hint}
Medevac, 2 Verwundete, Status rot.
\end{hint}
\paragraph*{Type Mark}
Type Mark beschreibt, ähnlich wie beim 9-Liner, die Markierung des Ziels. Im Falle von
Logistik und Medevac handelt es sich dabei aber um die angepeilte Landezone. Auch hier
sind wie bei CAS Rauch, Laser oder Lichtsignale eine Möglichkeit.
