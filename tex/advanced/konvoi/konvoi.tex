\pagebreak
\chapter{Konvoi}
	Wenn das Einsatzgebiet in weiter Ferne liegt, eine Verlegung mit einem Helikopter nicht möglich ist und die Wanderschuhe durchgelaufen sind, greift man auf seinen Fuhrpark zurück. Damit die Verlegung aber nicht in einer Offroad-Rally endet, befassen wir uns mit dem Konvoi und der Einbindung von Fahrzeugen außerhalb der Panzerwaffe in den infanteristischen Kampf. 

\subsection{Vorbereitungen für den Konvoi}
	Die Vorbereitung des Konvois umfasst mehrere Teilbereiche: Die Karten- und Routenarbeit, die Reihenfolge der Trupps innerhalb des Konvois, das Aufsitzen der Trupps auf die Fahrzeuge und das Herstellen der Konvoi-Formation.

\subsubsection{Karten- und Routenarbeit für den Konvoi}
	Die Kartenarbeit wird im TTT von der Zugführung (Trp. Grün)  übernommen. Trp. Grün bildet hierbei gleichzeitig das Führungsfahrzeug im Konvoi - Nicht zu verwechseln mit dem Fahrzeug, das an der Spitze den Konvoi anführt! \\
	Die Zugführung plant die Route nach verschiedenen Gesichtspunkten und richtet dafür Checkpoints (CP's) ein, die der Konvoi während der Fahrt zu passieren hat: 

	\begin{itemize}
		\item schnellstmögliche Annäherung an das Zielgebiet
		\item möglichst sichere Route für den Konvoi
		\item Ausnutzung von Gelände und Tageszeit zu unserem Vorteil
		\item CP's sind dabei so zu setzen, dass sie markante Objekte sind, die man schnell im Gelände ausmachen kann (Kirchen, Tankstellen etc.) und/oder die Möglichkeit zum Sammeln des Konvois bieten und/oder eine dem Gelände entsprechend gute Position für eine Overwatch ermöglichen und/oder einen Richtungswechsel des Konvois markieren (Kreuzungen)
		\item CP's sind nicht mit Haltepunkten gleichzusetzen! Wenn an einem CP gehalten werden soll wird das von der Zugführung befohlen!
	\end{itemize}

\subsubsection{Die Reihenfolge der Trupps im Konvoi}
	Die Zugführung passt die Reihenfolge der Trupps den Gegebenheiten und der Verfügbarkeit anderer motorisierter Elemente an.

\subsubsection{Das Aufsitzen auf Fahrzeugen}
	Gerüchteweise werden immer wieder Kameraden im Konvoi vergessen oder 30 Mann warten, bis der vermeintlich Abtrünnige (der schon im Fahrzeug sitzt) endlich einsteigt. \\
	Bestiegen werden Fahrzeuge in folgender Reihenfolge – von der höchsten Zahl zur niedrigsten (Truppführer). Dabei wird vor dem Einsteigen per Funk mitgeteilt, dass man einsteigt (>>6 steigt ein!<<, >>5 steigt ein!<< usw.). \\
	Der übliche Befehl eines Truppführers lautet hierbei >>Trupp XYZ, in umgekehrter Reihenfolge aufsitzen!<<.
Im Fahrzeug wird durchgezählt – der Truppführer beginnt. Nach Durchzählung gibt der Truppführer der Konvoi-Führung durch, dass der Trupp aufgesessen und bereit ist. \\
	Beim Absitzen steigt wiederum die höchste Zahl als Erstes aus – die niedrigste Zahl zuletzt (Truppführer). Somit kann der Truppführer davon ausgehen, dass tatsächlich sein gesamter Trupp abgesessen ist. Auch hier gibt man kurz durch, dass man absitzt (>>6 raus!<<, >>5 raus!<< usw.). \\

\subsubsection{Besetzung der Positionen im Fahrzeug}
	Führen, fahren, funken, funzt? \\
	Nein, tut es nicht! \\
	Deswegen fährt im TTT immer die größte Nummer, im Normalfall also Nummer 6. Truppführer, Funker und Medics haben striktes  Fahrverbot, sie haben andere Aufgaben (>>So, ich näh den jetzt hier mal fix zu …- Schau auf die Straße Grunt, schau auf die Stra...<< *BUMM*). \\
	Bei einem bewaffneten Fahrzeug wird die Position des Schützen im Normalfall von der Nummer 5 übernommen. Somit haben wir das Fahrzeug unter der Kontrolle eines Buddyteams, die anderen beiden Buddyteams werden nicht auseinander gerissen und sind voll einsatzfähig. Des Weiteren stellen wir so auch eine Feuerüberlegenheit her, denn die Nummern 3 und 4 bilden das MG-Team. Somit stehen dem Truppführer 2 Schwerpunktwaffen zu Verfügung (MG-Team + Fahrzeugwaffe). \\
	Der Truppführer nimmt hierbei die Position des Beifahrers ein, er hat ein GPS und kann den Fahrer einweisen. \\
	Idealerweise sieht also ein besetztes Fahrzeug im TTT wie folgt aus:

	\begin{itemize}
		\item Nummer 1 (Trp.Fhr.) - Beifahrer
		\item Nummer 2 (Grenadier) - Passagier
		\item Nummer 3 (MG-Assi) - Passagier
		\item Nummer 4 (MG) - Passagier
		\item Nummer 5 (AT/AA-Schütze) - Fahrzeugwaffenschütze
		\item Nummer 6 (AT/AA-Schütze) - Fahrer
	\end{itemize}

\subsubsection{Formation des Konvois}
	Wir fahren zwei unterschiedliche Konvoi-Formen:
	\begin{itemize}
		\item Offener Konvoi: Kolonne, Abstand von maximal 100 Metern (taugt für flaches Gelände, Wüsten, gute Wetterlage, guter Straßenbelag), hohe Geschwindigkeit.
		\item Geschlossener Konvoi: Kolonne, Abstand von unter 25-50 Metern (taugt für urbane Umgebung, schlecht einsehbares Gelände (Hügel, Bewuchs) hohes Verkehrsaufkommen, schlechte Wetterlage, schlechter Untergrund), niedrige Geschwindigkeit
	\end{itemize}

	ARMA-Reality-Check: Bei vielen Desyncs empfiehlt es sich einen höheren Abstand zwischen die Fahrzeuge zu legen. Auch mehr als 100 Meter. \\
	Generell sind die Abstände des Konvois der Topographie, der Sichtverhältnisse und der Feindlage anzupassen. Ein Auffahren auf unter 5 Meter (oder eine Fahrzeuglänge) ist unter allen Umständen zu vermeiden, außer explizit angeordnet. \\
	Dazu folgende Faustregeln:

	\begin{itemize}
		\item Je offener und höher die Sichtweite, umso größer kann der Abstand zwischen den Konvoi-Fahrzeugen gewählt werden. Trotzdem MUSS im Konvoi in der Regel Sichtkontakt zum vorderen Fahrzeug immer möglich sein.
		\item Je unübersichtlicher das Gelände, desto enger wird der Konvoi gefahren. Auch bei hohem Verkehrsaufkommen (zivile Fahrzeuge auf der Fahrbahn) wird eine eher enge Fahrweise bevorzugt.
	\end{itemize}

Vor- und Nachteile von dem offenen und geschlossenen Konvoi \\
	Generell ist eine lockere Formation in Takistan bevorzugen, da es sich meist um offenes Gebiet handelt. Die lockere Formation sorgt dafür, dass der Gegner nicht den gesamten Konvoi gleichzeitig unter Feuer nehmen kann und das IEDs nicht mehrere Fahrzeuge ansprengen können. Zudem ermöglicht der größere Abstand zwischen den Fahrzeugen mehr Manöverwege (Flanken, Ausweichen) und bessere Reaktion auf Gefahren. Nicht zuletzt reduziert es die Gefahr von Auffahrunfällen drastisch.  \\
	Die geschlossene Formation ist im Personenschutz und beim Transport von High Value Gütern bevorzugt, da die enge Formation verhindert, dass sich etwas >>zwischen<< den Konvoi setzen kann – und Feuerkraft auf einen sehr engen Raum zusammenfasst. Gleichzeitig ist die enge Formation deutlich schwieriger zu fahren (Auffahrunfälle) und erfordert vorausschauendes Fahren und ein eingespieltes Team. \\

\subsubsection{Struktur eines Konvois}
	Der Konvoi ist so strukturiert, dass er maximale Sicherung bietet. Dabei unterscheiden wir zwischen: \\
	Einem >>Dreier Konvoi<<(3 Elemente) und einem >>Fünfer Konvoi<< (5 Elemente).Egal wie groß die Mannschaft – ein Konvoi ist immer in mind. 3 Teile zu unterteilen – optimalerweise aber in fünf.  \\
Der >>Dreier Konvoi<<

	\begin{itemize}
		\item 1ste Sicherungselement
		\item Transport und Führung
		\item 2tes Sicherungselement
	\end{itemize}

	Die  Advanced Guard sorgt für die Sicherheit nach vorne und klärt die zu befahrende Wegstrecke auf. Die Transportgüter und die Führung befinden sich in dem mittleren Teil des Konvois und sichern die Seitenbereiche rechts und links ab. Die Nachhut sorgt für die rückwärtige Sicherung und gilt als schnelle Einsatztruppe bei Feuerüberfällen. \\
Der >>Fünfer Konvoi<<
Eine verbesserte Variante des Konvois im TTT besteht aus 5 Elementen:

	\begin{itemize}
		\item Recon (advance guard/Vorhut)
		\item 1tes Sicherungselement
		\item Transport und Führung
		\item 2tes Sicherungselement
		\item Nachhut (rear guard)
	\end{itemize}

	Durch diese Struktur übernehmen das 1ste und das 2te Sicherungselement die Seitenabsicherung und bei Bedarf ebenfalls die Sicherung nach vorne und hinten. Das Recon-Element kann den Abstand zum Konvoi vergrößern und wird als variables Aufklärungselement eingesetzt. Gefahren werden so frühzeitiger erkannt. Die Nachhut kann ebenfalls den Abstand vergrößern und so bei jedem Halt als Overwatch-Element dienen - sowie als Reserve-Einheit an Brennpunkte gezogen werden. \\

\subsection{Die Fahrt}
\subsubsection{Geschwindigkeit eines Konvois}
	Die Geschwindigkeit sollte in etwa 50-70 \% der Maximalgeschwindigkeit des LANGSAMSTEN Fahrzeuges im Konvoi sein. Jedes Fahrzeug braucht eine Beschleunigungsreserve für den Notfall und für das Regulieren des Abstandes. Das Fahren mit Höchstgeschwindigkeit ist nur im Notfall erlaubt und nur über kurze Strecken. \\
	Tipp: ARMA bietet 3 Standardgeschwindigkeiten für Fahrzeuge (tatsächliche ist Geschwindigkeit abhängig vom Fahrzeugmodell und Untergrund) \\

	\begin{itemize}
		\item Langsam: [Strg] + [W]
		\item Normal: [W]
		\item Schnell:[Shift] + [W]
	\end{itemize}

\subsubsection{Halten eines Konvois}
	Im TTT gibt es zwei relevanten Halteformationen für Konvois

	\begin{itemize}
		\item Halt in Kolonnenformation (-----)
		\item Halt in Fischgrätenformation (o<<<<)
	\end{itemize}

	Wichtig: Halten ist KEIN Parken. Halten bedeutet, dass jederzeit und sofort die Fahrt aufgenommen werden kann. Also keine >>für kleine Tiger<< für Fahrer während eines Haltes. \\
	Der Halt in Kolonnenformation wird auf der Straße vollzogen und die Fahrformation und Ordnung wird beibehalten. Zum Halt wird dabei an den Straßenrand gefahren (in der Regel rechts), um einen Rettungskorridor freizuhalten. \\
	Der Halt in Fischgrätenformation ist eine defensivere Formation als die Kolonnenformation und ermöglicht eine bessere Absicherung. Dabei wird das Fahrzeug von der Straße seitlich (rechts und links abwechselnd) im freien Gelände und in einem Winkel von etwa 45 Grad  zur Straße positioniert. Bei kurzem Halt, Abstand zur Straße etwa 5-10 Meter – und von Fahrzeug zu Fahrzeug in etwa 20-50 Meter. Bei längeren Halt wird die Sicherungselemente 100 Meter ins freie Feld gefahren und der Konvoi von dort aus gesichert. Würde man von oben auf den Konvoi sehen, entstünde die Form einer Fischgräte. \\
	Wann wird welche Formation angewendet? \\
	Die Fischgrätenformation ist die bessere, weil sicherere Formation. Allerdings erfordert sie deutlich mehr Platz, ein freies Feld rechts und links und gut zu befahrenden Untergrund. Wird auf ein schnelles Weiterfahren besonderen Wert gelegt oder es nicht möglich ist, die Straße zu verlassen (Mauern, Hindernisse, Brücke, abschüssiges Gelände) wird in Kolonnenformation gehalten. \\
\subsubsection{Sicherungsbereich in Fahrt}
Sicherungsbereiche während der Fahrt in 3er Formation

	\begin{itemize}
		\item Recon (advance guard) Sicherungsschwerpunkt nach vorne
		\item Transport und Führung -- Sicherungsschwerpunkt nach rechts und links
		\item Nachhut (rear guard) -- Sicherungsschwerpunkt nach hinten
	\end{itemize}

Sicherungsbereiche während der Fahrt in 5er Formation

	\begin{itemize}
		\item Recon (advance guard) Sicherungsschwerpunkt nach vorne
		\item 1tes Sicherungselement Sicherungsschwerpunkt nach rechts und links
		\item Transport und Führung  
		\item 2tes Sicherungselement Sicherungsschwerpunkt nach rechts und links
		\item Nachhut (rear guard) - Sicherungsschwerpunkt nach hinten
	\end{itemize}

\subsubsection{Sicherungsbereich in Fahrt – Personal}
	Während der Fahrt in einem bewaffneten Fahrzeug (HMMWV, MRAP, Fennek, Puma) gibt es in der Regel 3 Positionen zu besetzen.

	\begin{itemize}
		\item Fahrer (größte Nummer)
		\item Beifahrer/Kommandant (Truppführer) -> Wird entschieden
		\item Schütze (zweitgrößte Nummer)
	\end{itemize}

	In einem Fahrzeug mit Kommandantenposten (Fennek, MRAP Puma, etc.) wird die Wärmebild vom Truppführer besetzt, der Funker sitzt auf den Beifahrersitz vorne (Fennek, MRAP) und blickt nach vorne und rechts. \\
	In einem Fahrzeug OHNE Kommandantenposten (HMMWV, Toyota, LKW), wechselt der Truppführer auf den Beifahrersitz vorne. Der Truppführer hat die Aufgabe, den Sicherungsbereich vorne und rechts im Auge zu behalten. \\
	Bei Bewaffneten Fahrzeugen teilen sich die Fahrzeuge die Sicherungsbereiche nach Schwerpunkten auf. Das heißt, die Bordkanone wird auf den Schwerpunkt ausgerichtet. Dabei gilt der Merksatz >>rechts vor links<< – illustriert an folgendem Beispiel: 

	\begin{itemize}
		\item Fahrzeug 1 – Recon – Sicherung Schwerpunkt vorne
		\item Fahrzeug 2 – Sicherungselement 1 – Sicherungsschwerpunkt rechts
		\item Truppführung – Fahrzeug unbewaffnet
		\item Fahrzeug 4 – Sicherungselement 2 – Schwerpunkt links
		\item Fahrzeug 5 – Nachhut – Sicherung Schwerpunkt hinten
	\end{itemize}

	Bei Fahrzeugen mit Bewaffnung und Kommandantenposten ist darauf zu achten, dass beide unterschiedliche „Uhrzeiten“ überwachen. \\
	zweites Beispiel:

	\begin{itemize}
		\item Fahrzeug 1 – Recon – Sicherung Schwerpunkt vorne
		\item Fahrzeug 2 – Sicherungselement 1 – Sicherungsschwerpunkt rechts
		\item Fahrzeug 3 – Sicherungselement 1 – Sicherungsschwerpunkt links
		\item Fahrzeug 4 – Sicherungselement 1 – Sicherungsschwerpunkt rechts
		\item Fahrzeug 5 – Truppführung – Fahrzeug unbewaffnet
		\item Fahrzeug 6 – Sicherungselement 2 – Schwerpunkt links
		\item Fahrzeug 7 – Sicherungselement 2 – Sicherungsschwerpunkt rechts
		\item Fahrzeug 8 – Nachhut – Sicherung Schwerpunkt hinten
	\end{itemize}

\subsubsection{Sicherungsbereich im Halt}
	Hier muss klar unterschieden werden, zwischen den Methoden im  realen Leben und den Methoden in ARMA. Der Grund: Das Schadensmodell der Fahrzeuge in ARMA 3 ist (wie in allen Teilen zuvor) äußerst mangelhaft. Die Fahrzeuge tendieren ohne Vorwarndung in einen Feuerball aufzugehen. \\
	Was wieder nicht ganz so unrealistisch ist: Der Gegner (in diesem Falle die KI) kämpft mit Panzerbrechende Munition besonders erfolgreich Fahrzeuge - meist deutlich früher, als die AT-Schützen aufgeklärt wurden. Dies führt häufig dazu, dass bei >>realistischer<< Verhaltensweise der komplette Trupp im Feuerball stirbt. Daher gibt es, abweichend von den üblichen Doktrinen, Abweichungen im TTT 

	\begin{itemize}
		\item Es ist absolut zu vermeiden, die Fahrzeuge als Deckung zu verwenden. Das gilt sowohl für Softskin (leichte Fahrzeuge, nicht beschusssicher) als auch für Hardskin (gepanzerte Fahrzeuge)
		\item Die Sicherung von Fahrzeugen oder dem Konvoi findet IMMER in Deckung und IMMER etwa 15 Meter vom Fahrzeug entfernt statt - Ausnahme - das Fahrzeug ist nicht besetzt.
		\item Es kommt häufig vor, dass ein Trupp- oder Zugführer das komplette aussteigen anordnet, auch der Schütze und Fahrer. Das hängt damit zusammen, dass leere Fahrzeuge nicht angegriffen werden. Stattdessen werden ERST die Feinde aufgeklärt und dann bei Bedarf das Geschütz bemannt - wir verwenden also das Fahrzeug häufig als >>Joker<<  der erst gezogen wird, wenn wir das Gefährt sicher einsetzen können.
	\end{itemize}

Beim Halt gibt es zwei Arten des Absitzen - die Befehle:

	\begin{itemize}
		\item Absitzen (Alle steigen aus bis auf den Schützen - der Fahrer bleibt außen AM Fahrzeug in Rufweite zum Schützen - 5-10 Meter). Damit er sofort einsteigen kann.
		\item Verlassen (alle verlassen das Fahrzeug und lassen es stehen - der Trupp bewegt sich mind. 15 Meter vom Fahrzeug weg
	\end{itemize}

	Sicherungsrichtung beim Halt \\
Egal ob Kolonnenformation oder Fischgrätenformation: Im Konvoi bauen wir die Sicherung auf die Seite auf, die in unserem Verantwortungsbereich lag (bei der Fahrt). Also entweder Rechts oder Links. Dabei (wichtig!) entfernen wir uns von den Fahrzeugen und suchen eigenständig Deckung. Bei kleineren Konvois (3er Konvois) tragt der Mittelteil die Verantwortung für die seitliche Absicherung. Beim 3er Konvoi unbedingt zu vermeiden ist eine Absicherung, in der in der Mitte das Fahrzeug die Sicht auf den eigenen Trupp verhindert - und den Trupp in zwei Teile aufteilt. Besser ist es, sich vom Fahrzeug weg zu bewegen und an erhöhter Position in Deckung eine 360 Grad-Sicherung aufzubauen, mit Überblick auf den Konvoi. \\
	Die Recon richtet ihren Sicherungsbereich nach vorne aus. Das bewaffnete Fahrzeug ist nach Möglichkeit teilverdeckt nahe der Straße abzustellen - Aber NICHT auf Straßenkreuzungen oder ähnliches. \\
	Die Nachhut stoppt in Overwatchposition - dabei bleibt in der Regel das schwere MG besetzt, während ein Teiltrupp den Rückraum sichert. \\

\subsubsection{Konvoifunk und Kommunikation}
	VOR der Konvoi-Bildung wird vom Fahrer (!) und Truppführer (!) sich auf einen zusätzlichen Funkkanal eingewählt. Der Standard-Konvoi-Kanal im TTT ist 99 auf der Shortrange. Funker bleibt auf seinen Frequenzen. Nur Trupp- und Zugführer und OPL sprechen auf dem Kanal. Fahrer empfängt nur. Bei Ausfall vom Truppführer übernimmt Funker die Position vom TF. \\
	Über den Konvoikanal werden durchgegeben:

	\begin{itemize}
		\item Bewegung und Halt 
		\item Bedrohungslagen
	\end{itemize}

	Beispiel für einen Funkspruch bei plötzlichen Kontakt: Break, Break, Vorhut 1, heißer Kontakt, 3 Uhr. \\
	Alles andere läuft über die üblichen Kanäle. \\

\subsection{Reaktion auf verschiedene Bedrohungslagen}
in Arbeit
Battledrill: Durchstoßen
Kommt noch…
Battledrill: Bekämpfen
Kommt noch…
Battledrill: Ausweichen

