\pagebreak
\section{Close Air Support anfordern}
	Zu den Aufgabenspektrums einen Funkers im TTT gehört auch die Anforderung von Luftnahunterstützung, wie diese zu planen und umsetzten ist, wird im folgenden Kapitel behandelt.
\subsection{Die Plaungunsphase}
	Die Planungsphase ist der erste und wichtigste Schritt für erfolgreichen CAS. Der FAC verschafft sich einen Überblick über die zur Verfügung stehenden Mittel. Dabei überlegt er, was er genau erreichen will und wie er es am Besten umsetzen kann. Wichtige Faktoren dabei sind Positionen verbündeter oder ziviler Kräfte, sowie Einschränkungen durch Gelände oder Feind (z.\,B. Hügel, Flugabwehr, etc.).
	\par\medskip
	Die drei W's:\\
	\begin{tabular}{ll}
	\textbf{Wer} &  steht mir zur Verfügung?\\ 
	\textbf{Was} &  will ich erreichen?\\ 
	\textbf{Wie} &  kann ich es am Besten umsetzten?\\ 
	\end{tabular} 

\paragraph*{Check-in Briefing}
	\begin{quote}
		\glqq Wer steht mir zur Verfügung?\grqq
	\end{quote}
	Diese Frage lässt sich mit einem einfachen Protokoll abfertigen, dem >>Check-in Briefing<<. Dieses Protokoll wird in der Anfangsphase einer Mission, von Pilot und FAC, über Funk abgearbeitet. Der FAC hat dadurch eine genaue Vorstellung welche Arbeitsmittel ihm zur Verfügung stehen. Das Check-in Briefing besteht aus folgenden Informationen:
	\begin{itemize}
		\item Typ des CAS"=Objektes
		\item momentane Position
		\item Bewaffnung
		\item Verfügbarkeitszeitraum
	\end{itemize}

\paragraph*{Was will ich erreichen und wie kann ich es am Besten umsetzen?}
	Neben den neun namensgebenden Informationen des 9-Liners, muss auch der Typ des
	Angriffs bestimmt werden. Dabei wird zwischen drei Arten unterschieden.
	Bei einem Typ 1 Angriff, welcher in den meisten Fällen genutzt werden sollte, haben
	sowohl FAC, als auch das Flugobjekt Sicht auf das Ziel und zueinander.
	Der Typ 2 Angriff deckt mehrere Fälle ab. Hierbei können ein oder mehrere Optionen
	zutreffen:
	\begin{itemize}
		\item CAS sieht Ziel nicht
		\item FAC sieht Ziel nicht
		\item FAC sieht CAS nicht
	\end{itemize}
	Ein Typ 3 Angriff verhält sich von den Parametern ebenfalls wie der Typ 2 Angriff. Dabei
	darf das Flugobjekt Ziele eigenständig bekämpfen, welche sich in der vom FAC
	angegebenen Zielbereich befinden.\par
	Für die Kernplanung ist es wichtig, dass der FAC seine CAS"=Maschine kennt, damit er im
	Bereich des Möglichen plant. In die Planung des Angriffs spielen viele Faktoren ein, wie
	z.\,B. Anflugschneise, verbündete Kräfte, Bewaffnung, Feindtyp, etc. Hat der FAC die
	Planung des Angriffs beendet, gibt er diese in Form eines Protokolls über Funk durch.
	Dieses Protokoll wird auch >>9-Liner<< genannt. Das Fomular deckt alle wichtigen
	Informationen ab, damit der Pilot erfolgreich CAS fliegen kann. Folgende Informationen
	sind in einem 9-Liner enthalten.

\subsubsection{9-Liner}
	\begin{itemize}
		\item IP/BP
		\item Heading
		\item Distance
		\item Elevation
		\item Description
		\item Location
		\item Typ Mark
		\item Friendlies
		\item Egress
	\end{itemize}
	Hier wird noch einmal die Bedeutung der einzelnen Lines im Detail erklärt.
\paragraph*{IP\,/\,BP}
	Der Initial-Point ist der Ausgangspunkt des 9-Liners. Auf diesem Ausgangspunkt bauen
	alle weiteren Informationen auf. Das Flugobjekt fliegt zuerst diesen Punkt an und führt
	nach gegebenen Informationen den Einsatz aus.
	Die Battleposition ist eine Kampfzone für Helikopter, von der aus der Helikopter den CAS"=Auftrag ausführt.
\paragraph*{Heading}
	Der Heading bezeichnet die Sicht-, Bewegungsrichtung an der sich das CAS"=Objekt
	orientieren muss. Hierbei wird mit Gradzahlen gearbeitet. Der Offset ist dabei optional
	dieser wird z.\,B. bei besonderen Gefahrensituationen eingesetzt.
\paragraph*{Distance}
	Bezeichnet den Abstand, welche sich zwischen IP und Ziel befindet.
\paragraph*{Elevation}
	In der Line \textit{Elevation} wird die Höhe angegeben, auf der sich das Ziel befindet.
\paragraph*{Description}
	Bei der \textit{Description} handelt es sich um eine kurze Zielbeschreibung für den Piloten. Er kann sich dadurch besser auf sein zugewiesenes Ziel einstellen.
\paragraph*{Location}
	Die \textit{Location} ist das Grid, auf dem sich das Ziel befindet.
\paragraph*{Typ Mark}
	\textit{Type Mark} beschreibt die Markierung des Ziels. Möglichkeiten hierbei sind z.\,B. Rauch, Laser, oder keine Markierung.
\paragraph*{Friendlies}
	\textit{Friendlies} gibt an, ob und wo sich verbündete Kräfte aufhalten. Diese können sich optional auch mit z.\,B. grünem Rauch markieren. Ist der CAS"=Angriff mit >>Danger Close<< betitelt, wird die Markierung eigener Truppen zur Pflicht.
\paragraph*{Egress}
	Der Egress bestimmt die Abflugrichtung des Flugobjekts. Ist z.\,B. der Luftraum hinter dem Ziel nicht aufgeklärt, kann es unmöglich durch diesen Bereich abdrehen und zur IP zurückkehren.

\subsection{Die Aktionsphase}
	Nachdem der Liner über Funk an den Piloten übermittelt wurde, beginnt die Aktionsphase. Der Pilot beginnt mit der Ausführung der Befehle.\par
	Gehen wir von einem 9-Line-Briefing aus, bei dem der Pilot den 9-Liner bereits ausgewertet hat und bei der IP auf Standby steht. Ab der IP, dem Zentrum des Briefings beginnt der Angriff. Damit Pilot und FAC Anhaltspunkte über den Status des CAS haben, gibt es 3 kurze Meldungen an Schlüsselpunkten des Angriffs.\par
	Die erste Meldung gibt der Pilot bei Erreichen der IP durch:
	\begin{hint}
	Reaper inbound, kommen.
	\end{hint}
	Der FAC hat hier zwei Optionen. Entweder die Bestätigung zur weiteren Ausführung nach Plan oder eine letze Abbruchchance. Zum Bestätigen des Angriffs wird >>Continue<< und zum Abbruch des Angriffs >>Abort<< verwendet.\par
	Gehen wir davon aus der Pilot bekommt das >>Continue<<. Laut Liner wurde das Ziel mit rotem Rauch markiert. Für den Pilot bedeutet das, dass er sich sobald er Sichtkontakt zum Ziel hat, beim FAC mit >>Spot<< melden muss. Sollte er das Ziel nicht erkennen ist >>No Spot<< zu verwenden, wenn letzeres eintritt dreht der Pilot ab. Keinesfalls schießt er nach Vermutung.\par
	In unserem Beispiel gehen wir wieder davon aus, dass der Pilot >>Spot<< durchgibt. Ist er feuerbereit meldet er >>Tally<<. Darauf muss der FAC ihm schnellstmöglich >>Cleared Hot<< funken, da der Pilot noch keine Feuerfreigabe hat. Sollte der Pilot das Ziel erfasst haben, jedoch kein >>Cleared Hot<< hören, dann isSpott der Angriff abzubrechen und über die Abflugszone zurück zur IP zu fliegen. Kommt es vor, dass der Pilot das Ziel sieht, aber aus Gründen nicht feuerbereit ist, meldet er >>No Joy<<. Nach dem Angriff meldet der FAC je nach Erfolg >>Hit<< oder >>Miss<<, um den Piloten Feedback zu geben.

\subsection{Anwendung}
	Hier sind noch einmal zwei komplette CAS Beispiele hinterlegt. Die Theorie der letzten Kapitel wird hier angewendet und zum besseren Verständins mit Bildmaterial verdeutlicht.

\paragraph*{Check-In Briefing}
	Da sich noch keine Gelegenheit ergeben hat, klinkt sich der FAC auf die Frequenz von Reaper (Jet) und verlangt von ihm ein Check-in Briefing.
	\begin{longtable}{P{0.95\linewidth}}
	\toprule
	Grau meldet sich auf Funkfrequenz an, Reaper hier Grau, kommen.\\
	\rcg Reaper hört, kommen.\\
	Reaper, Anfrage auf Check-in Briefing, kommen.\\
	\rcg Reaper bestätigt, Check-in Briefing lautet wie folgt. 1x A10 Thunderbolt II. Altis Airport.
	Gau-8 30mm 1000x, GBU-12 4x, Hydras 14x, Maverickraketen 2x, Sidwinderraketen 2x,
	TOS 10 Minuten.\\
	Grau bestätigt Check-in Briefing, melde mich bei Bedarf, Ende.\\
	\bottomrule
	\end{longtable}

\paragraph*{Helikopter-CAS}
	Wir gehen von bereits abgeschlossenem Check-in Briefing aus. Grau hat einen Panzer
	aufgeklärt. Der Trupp will, dass Adler den Panzer mit einem Kampfsprung ausschaltet,
	dazu benutzen sie einen 9-Liner.\par
	\textit{Grau arbeitet den 9-Liner aus und klinkt sich auf die Frequenz von Adler. Der Trupp plant
	das Ziel mit Laser zu markieren, sodass Adler aufsteigen kann, den Laser aufschaltet und
	eine Hellfirerakete abfeuert, um dann sofort wieder abzutauchen.}
	\begin{longtable}{P{0.95\linewidth}}
	\toprule
	Grau meldet sich auf Funkfrequenz an, Adler hier Grau, kommen.\\
	\rcg Adler hört, kommen.\\
	Adler bereit für 9-Liner, kommen?\\
	\rcg Adler bestätigt, bereit für 9-Liner, kommen.\\
	Grau verstanden, 9-Liner lautet wie folgt. Typ 1 Angriff, BP Alpha, Heading SO xxx Grad,
	Distance x m, Description 1x MBT, Location xxx-xxx (Koordinate), Typ Mark Laser,
	Friendlies xxx m N, Egress NW Remarks, es wird mit Kampfsprüngen gearbeitet,
	lasergelenkter Raketenangriff. So verstanden, kommen?\\
	\rcg Adler bestätigt 9-Liner, Location xxx-xxx (Koordinate), Friendlies xxx m Nord, kommen.\\
	Grau bestätigt, melden wenn bei BP Alpha.\\
	\rcg \textit{Adler bewegt sich im Tiefflug zu BP Alpha, um nicht aufgeklärt zu werden. Nach Erreichen des BP meldet Adler sich wieder bei Grau.}\\
	\rcg Grau hier Adler, kommen.\\
	Grau hört, kommen.\\
	\rcg Adler bei BP Alpha, bereit für Kampfsprung, Laser on, kommen.\\
	Grau bestätigt, bereit für Kampfsprung, lasing, kommen. \textit{Grau schaltet den Laser an.}\\
	\rcg Adler bestätigt, lasing, Sprung auf Befehl.\\
	Hier Grau, Sprung!\\
	\rcg \textit{Adler führt einen Kampfsprung aus.}\\
	\rcg Spot!\\
	\rcg \textit{Adler erfasst den Laser.}\\
	\rcg Tally!\\
	Cleared Hot!\\
	\rcg Feuerfreigabe durch Grau, Adler schießt die Rakete ab. Einschlag der Rakete ins Ziel.\\
	Hit!\\
	\rcg Adler taucht ab.\\
	\rcg \textit{Adler taucht wie besprochen wieder hinter die Bergkuppe ab.}\\
	\bottomrule
	\end{longtable}

\paragraph*{Jet-CAS}
	Ausgangslage: Grau soll dem Zug bei der Einnahme einer Stadt zuarbeiten. 
	Der Auftrag ist die Deaktivierung eines Funkturms, um die Kommunikation des Feindes zu schädigen. Der Funkturm wird u.a. von einem schwer gepanzerten Fahrzeug bewacht. 
	Nach Anfrage hat Gelb Grau CAS Freigabe erteilt.
	\par\medskip
	Als Check-In Briefing wird das oben aufgeführte Beispiel genannt.\par
	Grau macht sich jetzt daran, den 9-Liner auszuarbeiten, um sich danach bei Reaper zu
	melden das wird überflüssig, sollte schon eine Taskforce gebildet sein. Nach
	abschließender Planung des FACs, meldet er sich wieder auf der Frequenz von Reaper.
	\begin{longtable}{P{0.95\linewidth}}
	\toprule
	Grau meldet sich auf Funkfrequenz an, Reaper hier Grau, kommen.\\
	\rcg Reaper hört, kommen.\\
	Reaper bereit für 9-Line Briefing, kommen?\\
	\rcg Reaper bestätigt. Bereit für 9-Line Briefing, kommen.\\
	Grau verstanden, 9-Line Briefing lautet wie folgt.Typ 2 Angriff, IP Romeo, Heading Süden xxx Grad, Distance x,x klicks, Elevation xx m, 1x MBT, Location xxx-xxx (Koordinate), Typ Mark roter Rauch,  Friendlies x, x klicks Nordwest, Egress NW, Remarks: Waffeneinschränkungen: GBU-12 So verstanden Reaper, kommen?\\
	\rcg Reaper bestätigt, Ziel auf xxx-xxx (Koordinate), Verbündete xxx m NW, kommen.\\
	Grau bestätigt, melden bei IP Romeo.\\
	\rcg \textit{Reaper wertet den 9-Liner aus und beginnt mit dem Anflug, jetzt beginnt die Aktionsphase. Zwei Minuten später, erreicht Reaper IP Romeo und meldet sich bei Grau.}\\
	\rcg Reaper inbound, kommen.\\
	Continue, lasing.\\
	\textit{Grau schaltet Laser ein}\\
	\rcg Spot. \textit{Reaper nimmt den Lasermaker auf den Radar war}\\
	\rcg Tally. \textit{Reaper ist unmittelbar vor dem Angriff und bekommt Sichtkontakt auf das Ziel bzw. den Leitlaser für die lasergelenkte Bombe.}\\
	Cleared Hot.\\
	\rcg \textit{Einschlag der Bombe, dass Ziel wurde getroffen}\\
	Hit, Standby auf IP Romeo, kommen.\\
	\rcg Bestätige Hit, Abflug xxx, Standby auf IP Romeo.\\
	\bottomrule
	\end{longtable}