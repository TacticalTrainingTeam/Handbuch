\pagebreak
\section{Close Quarters Battle}
\label{CQB}
\subsection{Technik: Zweier-Haken}
\begin{center}
	Modified Button hook
\end{center}
\begin{enumerate}
	\item Aufstellung Rechts und Links von der Tür (Türangeln beachten)
	\item Gleichzeitiger, überraschender Eintritt als Zwei-Mann-Team \\
		Feuerbereiche überlappen sich
	\item Zwei Mann laufen entlang der Wand, um etwaige Gegner in den Ecken zu bekämpfen
	\item Zwei Mann nehmen Sicherungsposition ein und bereiten sich auf weiteres Vorgehen vor
 \end{enumerate}        
 Vorteile
 \begin{itemize}
	 \item Sehr schnelles, einfaches Eindringen
	 \item Ideal für größere Räume oder Innenhöfe
	 \item Einfaches Manöver, auch für Anfänger geeignet
	 \item Überlappender Feuerbereich im Eingang
	 \item Für innen öffnende Türen ideal
 \end{itemize}     
  Nachteile
  \begin{itemize}
  	\item Tür / Eingang muss breit genug sein für nahezu gleichzeitiges Eindringen
  	\item Tür / Eingang muss vorher überquert werden (Überraschungseffekt)
  	\item Wenn die Tür sich nach außen öffnet, wird das gleichzeitige Eindringen erschwert. Team muss daher die Startposition anpassen
  \end{itemize}          
\subsection{Technik: 4-Mann-Zweier-Haken}
\begin{center}
	modified button hook -- Variante bei der Buddy-Teams erhalten bleiben
\end{center}
	\begin{enumerate}
		\item Aufstellung Rechts und Links von der Tür (Türangeln beachten) \\
		Klare Sektorenvergabe (links und rechts)
		\item Gleichzeitiger, überraschender Eintritt als 2-Mann-Team \\
		Feuerbereiche überlappen sich \\
		Eintrittsrichtung Beispiel: \\
		Nr.1 + 2: links; Nr. 3 + 4: rechts
		\item Vier Mann laufen entlang der Wand um etwaige Gegner in den Ecken zu bekämpfen
		\item Vier Mann nehmen Sicherungsposition ein und bereiten sich auf weiteres Vorgehen vor
	\end{enumerate}      
\subsection{Technik: 4-Mann-Fluten}
\subsection{Technik: Annähern an Gebäude -- Raupe}
\subsection{Technik: High-Low}
\subsection{Technik: Aufstellung Flashbang}